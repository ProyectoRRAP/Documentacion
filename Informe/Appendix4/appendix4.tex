% ******************************* Thesis Appendix C ********************************

\chapter{Archivos de configuración}

En el presente anexo se incluyen los scripts y archivos de configuraci\'on de las principales herramientas utilizadas en la construcción del prototipo.

Los siguientes archivos son tomados del PC identificado en el laboratorio de pruebas como \textbf{Galois}; los archivos de conifiguraci\'on de los restantes tres nodos son similares.

\section{Archivos de configuraci\'on Quagga}
Para la configuración de Quagga, se utilizan tres archivos: el archivo \textbf{daemons} en donde se indica la configuraci\'on de demonios (en el caso del proyecto se activa el demonio zebra y ospfd), el archivo \textbf{zebra.conf} y el archivo \textbf{ospfd.conf}.

\vspace{1cm}
Archivo \textbf{daemons}:
\begin{bash}
#
# Entries are in the format: <daemon>=(yes|no|priority)
#   0, "no"  = disabled
#   1, "yes" = highest priority
#   2 .. 10  = lower priorities
# Read /usr/share/doc/quagga/README.Debian for details.
#
# Sample configurations for these daemons can be found in
# /usr/share/doc/quagga/examples/.
#

vtysh=yes
zebra=yes
bgpd=no
ospfd=yes
ospf6d=no
ripd=no
ripngd=no
isisd=no

\end{bash}

\newpage

Archivo \textbf{zebra.conf}:
\begin{bash}
!
! Zebra configuration saved from vty
!   2015/02/05 21:26:43
!
hostname Router
password zebra
enable password zebra
log file /var/log/zebra.log
!
debug zebra kernel
!
!
interface eth0
 ipv6 nd suppress-ra
!
interface veth1
 ipv6 nd suppress-ra
!
interface lo
!
interface mpls0
 !ipv6 nd suppress-ra
!
interface vnf0
 ipv6 nd suppress-ra
!
interface vnf1
 ipv6 nd suppress-ra
!
interface vnf2
 ipv6 nd suppress-ra
!
interface vnf3
 ipv6 nd suppress-ra
!
ip forwarding
!
line vty
!
\end{bash}

\newpage
Archivo \textbf{ospfd.conf}:
\begin{bash}
! -*- ospf -*-
!
! OSPFd sample configuration file
!
!
hostname ospfd
password zebra
!enable password please-set-at-here
!
interface l0
!
interface eth0
  ip ospf cost 65535
!
interface veth1
 ip ospf cost 1
!
interface vnf0
  ip ospf cost 1
!
interface vnf1
  ip ospf cost 3
!
interface vnf2
  ip ospf cost 1
!
interface vnf3
  ip ospf cost 1
!

router ospf
 ospf router-id 192.168.1.11
 network 10.10.1.0/24 area 0.0.0.0
 network 10.10.4.0/24 area 0.0.0.0
 network 10.10.5.0/24 area 0.0.0.0
 network 192.168.1.0/24 area 0.0.0.0
 network 192.168.2.0/24 area 0.0.0.0
!
log stdout
!
\end{bash}

\newpage
\section{Archivos de configuraci\'on de Open vSwitch}

Script de configuraci\'on inicial de ovs:

\begin{bash}
###################################################
# Configuracion de Open vSwitch para nodo Galois  #
###################################################
#
# Declara bridge, interfaces
ovs-vsctl add-br bral
ovs-vsctl set bridge bral datapath_type=netdev
ovs-vsctl set bridge bral protocols=OpenFlow13
ovs-vsctl set bridge bral other-config:datapath-id=000000000000000B
ovs-vsctl add-port bral eth1
ovs-vsctl add-port bral nf0
ovs-vsctl add-port bral nf1
ovs-vsctl add-port bral nf2
#
# Setea por defecto el modo de falla como seguro
ovs-vsctl set-fail-mode bral secure
#
# Interfaces virtuales
ovs-vsctl add-port bral veth1 
-- set interface veth1 type=internal
ovs-vsctl add-port bral vnf0 
-- set interface vnf0 type=internal
ovs-vsctl add-port bral vnf1 
-- set interface vnf1 type=internal
ovs-vsctl add-port bral vnf2 
-- set interface vnf2 type=internal
#
# Conexion con el controlador
ovs-vsctl set-controller bral tcp:192.168.1.10:6633
#
\end{bash}


\newpage
Script para iniciar OVS:
\begin{bash}
ovsdb-server /usr/local/etc/openvswitch/conf.db \
--remote=punix:/usr/local/var/run/openvswitch/db.sock \
--remote=db:Open_vSwitch,Open_vSwitch,manager_options \
--private-key=db:Open_vSwitch,SSL,private_key \
--certificate=db:Open_vSwitch,SSL,certificate \
--bootstrap-ca-cert=db:Open_vSwitch,SSL,ca_cert \
--pidfile --detach 
--log-file=/var/log/openvswitch/openvswitch.log

ovs-vsctl --no-wait init
ovs-vswitchd --pidfile --detach 
--log-file=/var/log/openvswitch/vswitch.log
ovs-vsctl show

\end{bash}
\vspace{1cm}
Script de instalaci\'on de flujos para bridging entre interfaces f\'isicas y virtuales:
\begin{bash}
ovs-ofctl -O openflow13 add-flow bral table=0,priority=0,
hard_timeout=0,idle_timeout=0, in_port=1,actions=output:5
ovs-ofctl -O openflow13 add-flow bral table=0,priority=0,
hard_timeout=0,idle_timeout=0,in_port=5,actions=output:1

ovs-ofctl -O openflow13 add-flow bral table=0,priority=0,
hard_timeout=0,idle_timeout=0,in_port=2,actions=output:6
ovs-ofctl -O openflow13 add-flow bral table=0,priority=0,
hard_timeout=0,idle_timeout=0,in_port=6,actions=output:2

ovs-ofctl -O openflow13 add-flow bral table=0,priority=0,
hard_timeout=0,idle_timeout=0,in_port=3,actions=output:7
ovs-ofctl -O openflow13 add-flow bral table=0,priority=0,
hard_timeout=0,idle_timeout=0,in_port=7,actions=output:3

ovs-ofctl -O openflow13 add-flow bral table=0,priority=0,
hard_timeout=0,idle_timeout=0,in_port=4,actions=output:8
ovs-ofctl -O openflow13 add-flow bral table=0,priority=0,
hard_timeout=0,idle_timeout=0,in_port=8,actions=output:4
\end{bash}

\newpage
\section{Otros scripts de configuraci\'on}

Configuraci\'on de interfaces f\'isicas:
\begin{bash}

#######################################################
# Configuracion de interfaces de Red para Host Galois #
#######################################################
#
# Interfaz para la LAN de gestion
ifconfig eth0 192.168.1.11 netmask 255.255.255.0 up
#
# las interfaces fisicas no tienen direccion IP
ifconfig eth1 0.0.0.0 up
ifconfig nf0 0.0.0.0 up
ifconfig nf1 0.0.0.0 up
ifconfig nf2 0.0.0.0 up

\end{bash}
\vspace{1cm}
Configuraci\'on de interfaces virtuales:
\begin{bash}
# Interfaces virtuales Galois 
ifconfig veth1 192.168.2.11 netmask 255.255.255.0 up
ifconfig vnf0 10.10.1.1 netmask 255.255.255.0 up
ifconfig vnf1 10.10.5.1 netmask 255.255.255.0 up
ifconfig vnf2 10.10.4.1 netmask 255.255.255.0 up
\end{bash}
