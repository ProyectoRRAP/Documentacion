% ******************************************************************************
% ****************************** Custom Margin *********************************

% Add `custommargin' in the document class options to use this section
% Set {innerside margin / outerside margin / topmargin / bottom margin}  and
% other page dimensions
\ifsetCustomMargin
  \RequirePackage[left=37mm,right=30mm,top=35mm,bottom=30mm]{geometry}
  \setFancyHdr % To apply fancy header after geometry package is loaded
\fi

% *****************************************************************************
% ******************* Fonts (like different typewriter fonts etc.)*************

% Add `customfont' in the document class option to use this section

\ifsetCustomFont
  % Set your custom font here and use `customfont' in options. Leave empty to
  % load computer modern font (default LaTeX font).
  \RequirePackage{helvet}
\fi

% *****************************************************************************
% **************************** Custom Packages ********************************

% ************************* Algorithms and Pseudocode **************************

\usepackage{algpseudocode}
\usepackage[]{algorithm2e}

%\usepackage{algcompatible}
%\usepackage{algorithm}
%\usepackage{listings}


% ********************Captions and Hyperreferencing / URL **********************

% Captions: This makes captions of figures use a boldfaced small font.
%\RequirePackage[small,bf]{caption}

\RequirePackage[labelsep=space,tableposition=top]{caption}
\renewcommand{\figurename}{Fig.} %to support older versions of captions.sty


% *************************** Graphics and figures *****************************

%\usepackage{rotating}
%\usepackage{wrapfig}

% Uncomment the following two lines to force Latex to place the figure.
% Use [H] when including graphics. Note 'H' instead of 'h'
%\usepackage{float}
%\restylefloat{figure}

% Subcaption package is also available in the sty folder you can use that by
% uncommenting the following line
% This is for people stuck with older versions of texlive
%\usepackage{sty/caption/subcaption}
\usepackage{subcaption}

% ********************************** Tables ************************************
\usepackage{booktabs} % For professional looking tables
\usepackage{multirow}

%\usepackage{multicol}
%\usepackage{longtable}
%\usepackage{tabularx}


% ***************************** Math and SI Units ******************************

\usepackage{amsfonts}
\usepackage{amsmath}
\usepackage{amssymb}
\usepackage{siunitx} % use this package module for SI units


% ******************************* Line Spacing *********************************

% Choose linespacing as appropriate. Default is one-half line spacing as per the
% University guidelines

% \doublespacing
% \onehalfspacing
% \singlespacing


% ************************ Formatting / Footnote *******************************

% Don't break enumeration (etc.) across pages in an ugly manner (default 10000)
%\clubpenalty=500
%\widowpenalty=500

%\usepackage[perpage]{footmisc} %Range of footnote options


% *****************************************************************************
% *************************** Bibliography  and References ********************

%\usepackage{cleveref} %Referencing without need to explicitly state fig /table

% Add `custombib' in the document class option to use this section
\ifuseCustomBib
   %\RequirePackage[square, sort, numbers, authoryear]{natbib} % CustomBib

% If you would like to use biblatex for your reference management, as opposed to the default `natbibpackage` pass the option `custombib` in the document class. Comment out the previous line to make sure you don't load the natbib package. Uncomment the following lines and specify the location of references.bib file

\RequirePackage[backend=biber, style=numeric-comp, citestyle=numeric, sorting=none, natbib=false]{biblatex}
\bibliography{References/references} %Location of references.bib only for biblatex

\fi

% changes the default name `Bibliography` -> `References'
\renewcommand{\bibname}{Bibliografía}

% ------------ Glosario
\usepackage[refpages]{gloss}

\renewcommand*{\glossname}{Glosario}
%\newgloss{default}{.gls}{\glossname}{glsplain}
\makegloss
%\usepackage{glossaries}
%\usepackage[acronym]{glossaries}

\def\acknowledgementsname{acknowledgements}
\renewcommand{\acknowledgementsname}{Agradecimientos}

%\renewcommand*\abstractname{Summary}


%\usepackage[utf8]{}
%\usepackage[english]{babel}
%\addto\captionsenglish{\renewcommand*\abstractname{Summary}}


% *****************************************************************************
% *************** Changing the Visual Style of Chapter Headings ***************
% This section on visual style is from https://github.com/cambridge/thesis

% Uncomment the section below. Requires titlesec package.

%\RequirePackage{titlesec}
%\newcommand{\PreContentTitleFormat}{\titleformat{\chapter}[display]{\scshape\Large}
%{\Large\filleft{\chaptertitlename} \Huge\thechapter}
%{1ex}{}
%[\vspace{1ex}\titlerule]}
%\newcommand{\ContentTitleFormat}{\titleformat{\chapter}[display]{\scshape\huge}
%{\Large\filleft{\chaptertitlename} \Huge\thechapter}{1ex}
%{\titlerule\vspace{1ex}\filright}
%[\vspace{1ex}\titlerule]}
%\newcommand{\PostContentTitleFormat}{\PreContentTitleFormat}
%\PreContentTitleFormat

\renewcommand{\chaptername}{Capítulo}


% ******************************************************************************
% ************************* User Defined Commands ******************************
% ******************************************************************************

% *********** To change the name of Table of Contents / LOF and LOT ************

\renewcommand{\contentsname}{Índice general}
\renewcommand{\listfigurename}{Índice de figuras}
\renewcommand{\listtablename}{Índice de cuadros}


% ********************** TOC depth and numbering depth *************************

\setcounter{secnumdepth}{2}
\setcounter{tocdepth}{2}


% ******************************* Nomenclature *********************************

% To change the name of the Nomenclature section, uncomment the following line

%\renewcommand{\nomname}{Symbols}


% ********************************* Appendix ***********************************

% The default value of both \appendixtocname and \appendixpagename is `Appendices'. These names can all be changed via:

\renewcommand{\appendixtocname}{Lista de apendices}
\renewcommand{\appendixname}{Apendice}

% ******************************** Draft Mode **********************************

% Uncomment to disable figures in `draftmode'
%\setkeys{Gin}{draft=true}  % set draft to false to enable figures in `draft'

% These options are active only during the draft mode
% Default text is "Draft"
%\SetDraftText{DRAFT}

% Default Watermark location is top. Location (top/bottom)
%\SetDraftWMPosition{bottom}

% Draft Version - default is v1.0
%\SetDraftVersion{v1.1}

% Draft Text grayscale value (should be between 0-black and 1-white)
% Default value is 0.75
%\SetDraftGrayScale{0.8}


%% Todo notes functionality
%% Uncomment the following lines to have todonotes.

%\ifsetDraft
%	\usepackage[colorinlistoftodos]{todonotes}
%	\newcommand{\mynote}[1]{\todo[author=kks32,size=\small,inline,color=green!40]{#1}}
%\else
%	\newcommand{\mynote}[1]{}
%	\newcommand{\listoftodos}{}
%\fi

% Example todo: \mynote{Hey! I have a note}

\usepackage{tabularx}

% ************************* Algorithms and Pseudocode **************************
\usepackage{color}
\usepackage{listings}
\usepackage{setspace}

\definecolor{Code}{rgb}{0,0,0}
\definecolor{Decorators}{rgb}{0.5,0.5,0.5}
\definecolor{Numbers}{rgb}{0.5,0,0}
\definecolor{MatchingBrackets}{rgb}{0.25,0.5,0.5}
\definecolor{Keywords}{rgb}{0,0,1}
\definecolor{self}{rgb}{0,0,0}
\definecolor{Strings}{rgb}{0,0.63,0}
\definecolor{Comments}{rgb}{0,0.63,1}
\definecolor{Backquotes}{rgb}{0,0,0}
\definecolor{Classname}{rgb}{0,0,0}
\definecolor{FunctionName}{rgb}{0,0,0}
\definecolor{Operators}{rgb}{0,0,0}
%\definecolor{Background}{rgb}{0.98,0.98,0.98}
%\definecolor{BashBackground}{rgb}{0.18, 0.04, 0.14}
\definecolor{Background}{rgb}{0.96,0.96,0.96}
\definecolor{BashBackground}{rgb}{0.96,0.96,0.96}
\definecolor{BashText}{rgb}{0,0,0}

\lstnewenvironment{python}[1][]{
\lstset{
numbers=left,
numberstyle=\footnotesize,
numbersep=1em,
xleftmargin=1em,
framextopmargin=2em,
framexbottommargin=2em,
showspaces=false,
showtabs=false,
showstringspaces=false,
frame=l,
tabsize=4,
% Basic
basicstyle=\ttfamily\small\setstretch{1},
backgroundcolor=\color{Background},
language=Python,
% Comments
commentstyle=\color{Comments}\slshape,
% Strings
stringstyle=\color{Strings},
morecomment=[s][\color{Strings}]{"""}{"""},
morecomment=[s][\color{Strings}]{'''}{'''},
% keywords
morekeywords={import,from,class,def,for,while,if,is,in,elif,else,not,and,or,print,break,continue,return,True,False,None,access,as,,del,except,exec,finally,global,import,lambda,pass,print,raise,try,assert},
keywordstyle={\color{Keywords}\bfseries},
% additional keywords
morekeywords={[2]@invariant},
keywordstyle={[2]\color{Decorators}\slshape},
emph={self},
emphstyle={\color{self}\slshape},
%
}}{}


\lstnewenvironment{bash}[1][]{
\lstset{
numbers=left,
numberstyle=\footnotesize,
numbersep=1em,
xleftmargin=1em,
framextopmargin=2em,
framexbottommargin=2em,
showspaces=false,
showtabs=false,
showstringspaces=false,
%frame=l,
tabsize=4,
% Basic
basicstyle=\ttfamily\small\setstretch{1}\color{BashText},
backgroundcolor=\color{BashBackground},
language=bash,
% Comments
commentstyle=\color{BashText}\slshape,
% Strings
stringstyle=\color{BashText},
morecomment=[s][\color{BashText}]{"""}{"""},
morecomment=[s][\color{BashText}]{'''}{'''},
morecomment=[s][\color{BashText}]{"-"},
% keywords
morekeywords={\color{BashText}},
keywordstyle={\color{BashText}\slshape},
% additional keywords
morekeywords={[2]@invariant},
keywordstyle={[2]\color{BashText}},
emph={self},
emphstyle={\color{self}\slshape},
rulecolor=\color{BashText},
rulesepcolor=\color{BashText}
%
}}{}

% ************************* Creates Annex styles to add document **************************

% Create Roman Numeral Labelled Annexes
\newcommand{\annexname}{Anexo}
\makeatletter % treat @ as a letter instead of a control word.
\newcommand\annex{\par
\setcounter{chapter}{0}
\setcounter{section}{0}
\renewcommand\appendixname{Anexos}
\renewcommand\appendixpagename{Anexos}
\renewcommand{\appendixtocname}{Anexos}
\gdef\@chapapp{\annexname}
\gdef\thechapter{\@Roman\c@chapter}
\renewcommand{\theHchapter}{\annexname.\thechapter}
\addappheadtotoc
}
\makeatother


% ************************* General Styles ***********************************************
\renewcommand{\labelitemii}{$\circ$}


% ************************ Para hacer cuadraditos de colores *****************************
\usepackage{xcolor}
\newcommand\crule[3][black]{\textcolor{#1}{\rule{#2}{#3}}}


% ************************ Definicion de silabas *****************************************
\hyphenation{di-fe-ren-tes}
\hyphenation{he-rra-mien-ta}
\hyphenation{de-sa-rro-lla-da}
\hyphenation{pru-e-bas}
\hyphenation{de-sa-rro-llo}
\hyphenation{sig-ni-fi-ca-ti-va-men-te}
\hyphenation{im-ple-men-ta-ción}
\hyphenation{li-bre-men-te}
\hyphenation{ca-rac-te-rís-ti-cas}
\hyphenation{de-no-mi-na-do}
\hyphenation{he-rra-mien-ta}
\hyphenation{hard-ware}
\hyphenation{Net-FPGA}
\hyphenation{pro-duc-tos}
\hyphenation{Re-fe-ren-ce-NIC}
\hyphenation{li-cen-cias}
\hyphenation{per-fi-la}
\hyphenation{pro-ba-ble-men-te}
\hyphenation{u-ti-li-zar}
\hyphenation{Open-Flow}
\hyphenation{prin-ci-pal-men-te}
\hyphenation{res-tric-cio-nes}
\hyphenation{he-rra-mien-ta}
\hyphenation{co-mer-cia-li-za-ción}
\hyphenation{do-cu-men-ta-ción}
\hyphenation{co-rres-pon-dien-tes}

