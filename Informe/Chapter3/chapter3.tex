\chapter{An\'alisis del problema}

% **************************** Define Graphics Path **************************
\ifpdf
    \graphicspath{{Chapter3/Figs/Raster/}{Chapter3/Figs/PDF/}{Chapter3/Figs/}}
\else
    \graphicspath{{Chapter3/Figs/Vector/}{Chapter3/Figs/}}
\fi

En el abordaje de la problem\'atica planteada, surgen las primeras interrogantes acerca de la realidad a modelar, acerca de las tecnolog\'ias a utilizar, y en particular acerca del dise\~no en general de la soluci\'on. Algunas de estas interrogantes son ¿Como se implementan usualmente servicios de redes privadas virtuales(VPN)?, ¿Que arquitectura basada en SDN utilizar para el prototipo?, ¿Como constru\'ir un router opensource SDN a partir del hardware NetFPGA?, entre otras. 

Las respuestas a estas preguntas, establecen los lineamientos generales sobre los que se acenta el dise\~no del prototipo. Por ello destinamos el presente cap\'itulo al planteo de las principales interrogantes tanto conceptuales, como tecnol\'ogicas, como de dise\~no, que impactan en la soluci\'on.

\section[Como implementar servicios de redes privadas virtuales(VPN)]{Como implementar servicios de redes privadas virtuales (VPN)}

Existen diferentes formas de implementar sobre una infraestructura de red espec\'ifica, servicios de redes privadas virtuales o VPNs. En particular nos interesan aquellas que pueden garantizar ciertas caracter\'isticas sobre la calidad de los servicios brindados; lo que se conoce usualmente como calidad de servicios o QoS por su sigla en ingles.\\

En Uruguay, la empresa de telecomunicaciones Antel, ofrece dierentes productos orientados a brindar servicios de redes privadas virtuales. Estas implementaciones van desde VPNs Lan to Lan a trav\'es de  conexiones Ethernet Punto a Punto, pasando por servicios VLAN Hub tambi\'en implementado con conexiones Ethernet, hasta servicios de VPN sobre MPLS. 

De estas alternativas, la implementaci\'on de VPNs utilizando MPLS parece ser la opci\'on m\'as apropiada para el prototipo. Por un lado es una de las implementaciones utilizadas por Antel, mayor proveedor de servicios de telecomunicaciones en Uruguay. A su vez, actualmente MPLS parecer\'ia ser una de las tecnolog\'ias m\'as utilizadas a nivel mundial para proveer de servicios de VPNs con calidad de servicios. Por otro lado, MPLS permite incorporar QoS con los que se pueden implementar diferentes clases de servicios como se pretende en el nuevo prototipo de la RAU.\\

Bas\'andose en lo anterior, se puede decir que MPLS es una respuesta v\'alida a la pregunta de como implementar servicios de redes privadas virtuales. 

\section[Arquitectura MPLS pura o IP/MPLS h\'ibrida]{Arquitectura MPLS pura o IP/MPLS h\'ibrida}

[ACA HAY QUE HABLAR DE PORQUE CONVIENE USAR IP TAMBIEN Y QUE NO SEA PURAMENTE MPLS]
\section[Elecci\'on de un algor\'itmo de ruteo]{Elecci\'on de un algor\'itmo de ruteo}

\section[Algor\'itmo para la distribusi\'on de etiquetas]{Algor\'itmo para la distribusi\'on de etiquetas}

 
\section[Elecci\'on de una arquitectura SDN]{Elecci\'on de una arquitectura SDN}

%En el cap\'itulo dedicado al estado del arte de las redes definidas por software, se explica en profundidad la arquitectura de SDN, y a partir del mismo queda completamente sanjada la diferencia entre los conceptos de SDN y OpenFlow por ejemplo.\\

%Entonces, la primer decisi\'on de dise\~no importante a tener en cuenta es por cual implementaci\'on del enfoque SDN optar, puesto que las alternativas existentes son bien diferentes.\\
 
Naturalmente, otra de las decisiones de dise\~no importantes a tomar, es la elecci\'on de una de las arquitecturas existentes basadas en el enfoque SDN.\\
  
Dentro de las alternativas existentes, OpenFlow se presenta como una opci\'on madura y probada~\citep{Ofelia}. Como se menciono en el estado del arte OpenFlow  se ha caracterizado por un desarrollo sostenido y una amplia adopci\'on tanto por la academia como por la industria, adem\'as de ser compatible con una amplia variedad de tecnolog\'ias. Debido a esto cuenta con una amplia comunidad de usuarios, extensa documentaci\'on y es soportado por varias soluciones comerciales[referencia a switch hp, carajo carjao]. Finalmente se puede agregar que OpenFlow esta integramente desarrollado bajo la filosof\'ia opensource. 

Por estas razones, se decide por OpenFlow como la implementaci\'on del enfoque SDN a utilizar para el desarrollo del prototipo.\\

%[METER EN EL ESTADO DEL ARTE PARA JUSTIFICAR]
%nace como idea en el a\~no 2008, y luego de varias versiones de desarrollo libera su primera versi\'on oficial(OpenFlow v1.0.0) a finales del a\~no 2009. Tras ello se ha caracterizado por un desarrollo sostenido, siendo actualmente su versi\'on oficial la 1.5.
%[FIN]




%Por otro lado en las sucesivas versiones de OpenFlow, se han incorporado diferentes caracter\'isticas y funcionalidades que hacen en la actualidad a la versi\'on 1.5 de OpenFlow, una alternativa vers\'atil, flexible y completa.\\

Tomada la decisi\'on de utilizar OpenFlow, resta decidir en relaci\'on a esto que versi\'on utilizar.\\

Al momento de culminar el desarrollo de este trabajo, la versi\'on m\'as reciente de OpenFlow era la 1.5, mientras que al momento de iniciarse este proyecto la versi\'on m\'as reciente era la 1.4.

Por otro lado, la versi\'on de OpenFlow elegida debe garantizar ciertos requerimientos de diseni\~no a los que el prototipo esta sujeto. En particular interesa que garantice:

\begin{itemize}
\item Soporte para MPLS, tanto en la capacidad de reconocer los cabezales MPLS como para la manipulaci\'on de los mismos mediante las primitivas POP, PUSH y SWAP
\item Soporte para tags de VLAN (esta ultima no se si ponerla)
\end{itemize}

Sobre la primer restricci\'on, OpenFlow brinda soporte completo para MPLS a partir de la versi\'on 1.3. Por otro lado en relaci\'on a la segunda restricci\'on [COMPLETAR].\\

De esta forma, la versi\'on m\'as b\'asica de OpenFlow que asegura soporte a las restricciones de dise\~no mencionadas es la versi\'on 1.3.\\

M\'as adelante veremos en la siguiente secci\'on, la cual  esta destinada a la programaci\'on de la placa NetFPGA, que es de inter\'es trabajar con la versi\'on de OpenFlow m\'as sencilla y minimalista posible, que de soprte a todas las funcionalidades y restricciones impuestas sobre el protot\'ipo.\\ 

En conclusi\'on la versi\'on de OpenFlow a utilizar es la 1.3.

\section[Alternativas de dise\~no para el router]{Alternativas de dise\~no para el router}

Como se mencion\'o anteriormente, una de las premisas para la elaboraci\'on del router opensource es la utilizaci\'on del hardware NetFPGA. A su vez, partiendo de este hardware es necesario llegar a un dispositivo compatible con OpenFlow.\\ 

Existen dos estrategias bien definidas para la construcci\'on un switch OpenFlow, partiendo del hardware mencionado, y utilizando los diferentes proyectos disponibles para la programaci\'on del mismo. Una de ellas es programar el hardware para que se comporte como un switch compatible con el protocolo OpenFlow. La otra alternativa es programar el hardware para que se comporte como una tarjeta de red estandar, e implementar todo el comportamiento de un switch OpenFlow en software.\\

Para la primer estrat\'egia se cuenta con un proyecto desarrollado previamente para la plataforma NetFPGA, y disponible libremente en el repositorio de software de dicho producto. No obstante este proyecto presenta una dificultad y es que a pesar de haber sido dise\~ado para soportar en un futuro cualquier versi\'on disponible del protocolo OpenFlow, en su veri\'on actual solamente soporta un conjunto reducido de funcionalidades de la versi\'on 1.0 de este protocolo. \\
Como se mencion\'o anteriormente la versi\'on m\'as b\'asica de este protocolo que permite soportar el conjunto de funcionalidades relevadas como requerimientos para este proyecto es la versi\'on 1.3. Esto conlleva a la ncesidad de extender el proyecto existente, para soprtar las nuevas caracter\'isticas incorporadas en las sucesivas versiones posteriores a la 1.0; o al menos aquellas que son escenciales para soportar las caracter\'isticas pretendidas sobre el prototipo.\\

Para la segunda estrat\'egia se cuenta con un proyecto tambi\'en desarrollado previamente para la plataforma NetFPGA, llamado ReferenceNIC. Este proyecto habilita a programar el hardware para que se comporte como una placa de red estandar. Adicionalmente se debe incluir o desarrollar herramientas que permitan implementar por software el comportamiento de un switch OpenFlow. En particular sobre este \'ultimo punto vale la pena destacar la existencia de Open vSwitch, herramienta que entre otras caracter\'isticas realiza esto mismo, utilizando hardware tradicional como una placa de red estandar.

Contextualizando ambas alternativas en el marco de la realizaci\'on de este proyecto de fin de carrera, ambas alternativas presentan sus ventajas y desventajas; no siendo ninguna de ellas a priori mejor que la otra. Por ello a continuaci\'on se exponen comparativamente las principales ventajas y desventajas de cada alternativa, a modo de sustentar la elecci\'on de una ellas. 

\newpage
%%% Tabla Ventajas y Desventajas de ambas alternativas
\begin{table}[!Ht]\centering\small
\begin{tabularx}{\textwidth}{|>{\setlength\hsize{1.0\hsize}\setlength\linewidth{\hsize}}X|>{\setlength\hsize{1.0\hsize}\setlength\linewidth{\hsize}}X|}
\hline
\multicolumn{2}{|c|}{Ventajas}\\ \hline 
\hline
Extender proyecto OpenFlow NetFPGA & ReferenceNIC + Open vSwitch\\
\hline
\begin{itemize}
\item \'Optimo aprovechamiento de la capacidad de c\'omputo y procesamiento del hardware disponible.
\item Mayor posibilidad de lograr velocidades de procesamiento competitivas con productos comerciales similares.
\item Mayor posibilidad de obtener resultados aceptables, en performance y rendimiento para puesta en producción.

\end{itemize}
&
\begin{itemize}
\item No se tiene la necesidad de modificar o desarrollar software en el lenguaje y en el entorno de programaci\'on de la tarjeta NetFPGA.

\item Evitar desarrollar software para la NetFPGA ahorra tiempo de proyecto que se puede invertir en otras l\'ineas de trabajo, igualmente importantes. 

\item Programar el harware NetFPGA con proyectos precompilados como el ReferenceNIC requiere \'unicamente de licencias de software que son accesibles sin costo ya sea mediante licencias gratuitas o de prueba.
\end{itemize}
\\
\hline
\end{tabularx}
\end{table}

\begin{table}[!HT]\centering\small
\begin{tabularx}{\textwidth}{|>{\setlength\hsize{1.0\hsize}\setlength\linewidth{\hsize}}X|>{\setlength\hsize{1.0\hsize}\setlength\linewidth{\hsize}}X|}
\hline
\multicolumn{2}{|c|}{Desventajas}\\ \hline
\hline
Extender proyecto OpenFlow NetFPGA & ReferenceNIC + Open vSwitch\\
\hline
\begin{itemize}

\item Extender el proyecto existente, en si mismo constituye un empresa del porte de un proyecto de fin de carrera
\item El conocimiento técnico necesario se perfila m\'as al de un Ingeniero Eléctrico que al de un Ingeniero en Computación, lo cual constituye un riesgo del proyecto.
\item Desarrollar software para el hardware NetFPGA y compilarlo requiere de licencias de software costosas.
\end{itemize}

&

\begin{itemize}
\item No se aprovecha de forma óptima las capacidades de procesamiento del hardware disponible. En otras palabras se tiene hardwre ``caro'' y potente en forma ociosa.
\item Los resultados obtenidos en relaci\'on al rendimiento del prototipo, muy probablemente no sean los esperados para una equipo de producción.
\end{itemize}
\\
\hline
\end{tabularx}
\end{table}

\clearpage
\newpage
Habiendo presentado las principales ventajas y desventajas de cada alternativa, y teniendo presente el alcance y el tiempo disponible para la ejecuci\'on de este proyecto, se opt\'o por la segunda estrat\'egia presentada.\\

Optando por la segunda estrat\'egia presentada se logra obtener de forma temprana un prototipo de switch OpenFlow con el cual trabajar en la programaci\'on del datapath mediante un controlador, desarrollar estrategias para constru\'ir servicios en una red h\'ibrida IP/MPLS, as\'i como dise\~nar pruebas y un testbed acorde para validar tecnol\'ogicamente la soluci\'on propuesta.\\ 


\section[Alternativas de dise\~nio]{Plano de Control centralizado vs distribu\'ido}

%\section[Dise\~no general del prototipo]{Dise\~no general del prototipo}



