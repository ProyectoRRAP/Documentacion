\chapter{An\'alisis del problema}

% **************************** Define Graphics Path **************************
\ifpdf
    \graphicspath{{Chapter3/Figs/Raster/}{Chapter3/Figs/PDF/}{Chapter3/Figs/}}
\else
    \graphicspath{{Chapter3/Figs/Vector/}{Chapter3/Figs/}}
\fi

El primer paso en el proceso de construcción del prototipo para la RAU2, es el análisis del problema planteado. Definir en función de requerimientos el alcance del prototipo, investigar qu\'e alternativas se tienen para construir un nodo del prototipo a partir del hardware NetFPGA, resolver qu\'e estrategia ser\'a la utilizada para implementar servicios de VPNs, entre otros aspectos.

El presente cap\'itulo est\'a destinado al planteo de estos aspectos y los fundamentos sobre los que se basan las decisiones de diseño. 

\section[Definición de requerimientos]{Definición de requerimientos}
\label{3.1}

Para identificar posibles requerimientos a implementar en el prototipo para la RAU2, se trabaja inicialmente en identificar requerimientos generales sobre esta red. Luego los resultados obtenidos ser\'an contextualizados en el alcance, los objetivos y resultados esperados de este trabajo, alcanzando as\'i los requerimientos para el prototipo.

La RAU definió las funcionalidades esperadas para su evolución hacia RAU2, considerando a ANTEL como el proveedor de la infraestructura de red necesaria para el despliegue. Las principales características que se desean implementar son las siguientes:

\begin{enumerate}
\item \textbf{Clasificación y separación del tráfico:} Una de las necesidades y objetivos de la futura actualizaci\'on de la RAU, es la facilidad para clasificar y aislar tráfico. Alineado con las actuales necesidades, en particular se precisa al menos diferenciar las siguientes 3 categorías, as\'i como aislar el tr\'afico de cada categoría y para cada organizaci\'on: (a) público, (b) académico y (c) servicios de contenido.

\item \textbf{Manejo de grandes volúmenes de datos:} En la RAU intervienen instituciones como el Instituto Pasteur, Centro Uruguayo de Imagenología Molecular (CUDIM) entre otros, en donde la generación e intercambio de grandes volúmenes de datos como lo son los exámenes PET del Pasteur o una secuenciación de ADN del CUDIM entre otros es una de los servicios que la RAU2 deber\'a soportar.

\item \textbf{Escalabilidad:} Se espera alcanzar en un mediano plazo un total de 11.000 docentes, 7.000 funcionarios y 140.000 estudiantes, por lo que el prototipo para la RAU2 debe ser escalable en la cantidad de usuarios soportados por la infraestructura.

\item \textbf{Red de entrega de contenidos (Content Delivery Network):} Resulta sumamente útil tomar un enfoque de red de entrega de contenidos para el diseño de la red académica. Las organizaciones partícipes de la misma tienen y generan grandes volúmenes de información de gran interés por parte de otras organizaciones de la RAU. Una red de distribución de contenidos garantiza un mejor acceso en tiempo real a dicha información por parte de múltiples organizaciones en simultáneo. 
 
\end{enumerate}

Este conjunto de funcionalidades o requerimientos representa un problema de dimensiones mayores al que se pretende resolver en este trabajo, por ello pensando en resolver un problema acorde a las dimensiones de un proyecto de fin de carrera, se decide hacer foco en el primer requerimiento, la Clasificación de Tr\'afico.

Partiendo de la clasificación de tr\'afico como funcionalidad principal, se llega a los siguientes requerimientos:

\begin{table}[h!]\centering
\begin{tabularx}{\textwidth}{|>{\setlength\hsize{1.0\hsize}\setlength\linewidth{\hsize}}X|}
\hline
\multicolumn{1}{|c|}{Requerimientos Funcionales}\\ 
\hline
\begin{itemize}
\item Dada una organización o aplicación, el Sistema debe tener la capacidad de clasificar y distinguir el tráfico asociado a la misma de cualquier otro tipo de tráfico. Esto se conoce como clasificación de tráfico.

%\item Dado el tipo de tr\'afico asociado a una organización, el Sistema debe poder asignar un porcentaje de los recursos disponibles de la red para el procesamiento del mismo.

%\item Dado el tipo de tr\'afico asociado a una organización, el Sistema debe poder establecer m\'as de un camino en la red para transportar dicho tr\'afico implementando lo que se conoce como balanceo de carga.

\item Dadas dos subredes asociadas a una misma organización e interconectadas mediante el Sistema, se debe garantizar que la  numeración IP del tráfico generado por una de las subredes sea mantenida al ser entregado a la segunda subred, manteniendo de esta forma la identidad de los usuarios en ambas subredes. 
\end{itemize}\\
\hline
\end{tabularx}
\end{table}

\begin{table}[h!]\centering
\begin{tabularx}{\textwidth}{|>{\setlength\hsize{1.0\hsize}\setlength\linewidth{\hsize}}X|}
\hline
\multicolumn{1}{|c|}{Requerimientos no Funcionales}\\ 
\hline
\begin{itemize}

\item Open Source: En la medida que sea posible interesa utilizar herramientas y componentes libres y abiertas como software libre y de código abierto, hardware libre.

\end{itemize}\\
\hline
\end{tabularx}
\end{table}
  
\section[¿Por qu\'e utilizar SDN?]{¿Por qu\'e utilizar SDN?}

Se tiene ya una noción básica del sistema que se pretende construir, interesa entonces analizar por qu\'e se quiere utilizar el enfoque de SDN en el mismo.

El mecanismo estándar y transparente que propone SDN para manipular los diferentes dispositivos de red (se han hecho grandes esfuerzos para estandarizar algunas de las implementaciones de la Interfaz Sur existentes, OpenFlow es un buen ejemplo de ellos) brinda flexibilidad y agilidad para el desarrollo de nuevos protocolos y servicios de redes en una infraestructura de dispositivos heterogénea, facilitando as\'i la innovaci\'on en el \'area. A su vez, se gana independencia de la implementaci\'on propuesta por cada fabricante. Actualmente utilizar productos de un determinado fabricante para la construcci\'on de servicios, muchas veces implica configurar equipos y desarrollar aplicaciones sobre la plataforma del mismo, la cual usualmente tiene su propia API de operaciones generalmente propietaria y su propio lenguaje de desarrollo. Esto implica dificultades para la incorporaci\'on de nuevos dispositivos a una red existente, eventualmente de diferente fabricante, dificultades para la incorporaci\'on de nuevos servicios y m\'as dificultades para migrar servicios existentes, eventualmente a una plataforma diferente.

En pocas palabras, SDN facilita notablemente la investigaci\'on e innovaci\'on en el desarrollo de nuevos servicios y protocolos, eliminando barreras en la manipulaci\'on de los diferentes dispositivos de red, y ganando libertad de los fabricantes de hardware.

\section[¿Por qu\'e utilizar NetFPGA?]{¿Por qu\'e utilizar NetFPGA?}

Se necesita para el desarrollo de este proyecto una plataforma tecnol\'ogica compatible con SDN y en particular con el protocolo OpenFlow con la cual desarrollar un prototipo. El hardware NetFPGA puede configurarse para construir dicha plataforma a un precio significativamente menor que el de un equipo comercial, con funcionalidades y capacidades suficientes para el desarrollo del prototipo. 

Por otro lado, como se quiere estudiar la aplicabilidad del enfoque OpenFlow/SDN en la construcción de la RAU2, es necesario adem\'as de la construcción de un prototipo implementar algunos casos de uso representativos sobre el mismo en una red experimental. Por tanto, es imperativo la construcción de un laboratorio de pruebas compuesto por un conjunto de nodos que de ser hardware comercial tendría un costo significativamente mayor que el hardware NetFPGA.

Otro de los beneficios de utilizar hardware reconfigurable, es la capacidad de prototipar diferentes dispositivos de red en un mismo equipo. Se puede modificar din\'amicamente el laboratorio de pruebas en función de los casos de uso que se quieran probar sin la necesidad de adquirir nuevo hardware, reutilizando el disponible. Adem\'as la capacidad para programar el hardware, permite sacar provecho de las características del mismo para la implementaci\'on de funcionalidades espec\'ificas para la RAU, que en hardware comercial solo se podrían implementar si el mismo ya tuviera incorporadas dichas funcionalidades.

Finalmente la utilizaci\'on de hardware abierto, en conjunci\'on con software libre y de código abierto posibilitan la construcci\'on de lo que se podría llamar “Router Open Source”, un nodo para la nueva red académica basado en tecnologías abiertas. Esto otorgaría una mayor flexibilidad y capacidades de reutilizaci\'on del hardware disponible en la red académica avanzada.

\section[¿Qu\'e arquitectura basada en SDN utilizar?]{¿Qu\'e arquitectura basada en SDN utilizar?}
 
Naturalmente, otra de las decisiones de dise\~no importantes a tomar es la elecci\'on de una de las arquitecturas existentes basadas en el enfoque SDN. Dentro de las alternativas existentes, OpenFlow se presenta como una opci\'on madura y probada~\citep{Ofelia}. Como se mencion\'o en el estado del arte OpenFlow  se ha caracterizado por un desarrollo sostenido y una amplia adopci\'on tanto por la academia como por la industria, adem\'as de ser compatible con una amplia variedad de tecnolog\'ias. Debido a esto cuenta con una amplia comunidad de usuarios, extensa documentaci\'on y es soportado en varios productos comerciales (algunos ejemplos son \citep{Pica8}, \citep{HP}, \citep{Centec} y
 \citep{SDNProductlist}. Finalmente se puede agregar que OpenFlow est\'a íntegramente desarrollado bajo la filosof\'ia Open Source (c\'odigo abierto). 

Por estas razones, se decide utilizar OpenFlow como la implementaci\'on de SDN en el desarrollo del prototipo.

Por otro lado, OpenFlow ha ido incorporando nuevas funcionalidades a medida que evolucionaron las versiones del protocolo. Actualmente con cinco versiones estables entre las cuales se pueden destacar la versi\'on 1.4 (versi\'on disponible m\'as reciente al momento de iniciar este trabajo) y versi\'on 1.5 (versi\'on m\'as reciente al momento de culminar este trabajo). Se debe decidir entonces cu\'al es la versi\'on del protocolo a utilizar, teniendo en cuenta a su vez que la versi\'on elegida debe garantizar:

\begin{itemize}
\item Soporte para MPLS, tanto en la capacidad de reconocer los cabezales como para la manipulaci\'on de los mismos mediante las primitivas POP, PUSH y SWAP
\item Soporte para calidad de servicios (QoS)
\end{itemize}

A partir de la versión 1.3.1 OpenFlow brinda tanto soporte completo para MPLS, así como también incorpora el concepto de métricas por flujo, orientado a implementar funcionalidades de QoS. Por ello se decide utilizar la versión 1.3.1 pese a no ser la m\'as nueva para la construcción del prototipo.
 
En la siguiente secci\'on veremos por qu\'e es de inter\'es trabajar con la versi\'on de OpenFlow m\'as sencilla y minimalista posible que d\'e soporte a todas las funcionalidades y restricciones impuestas sobre el prototipo.

\section[Alternativas de dise\~no para el router]{Alternativas de dise\~no para el router}

Como se menciona en el cap\'itulo 1, una de las premisas para la construcción del prototipo es la utilizaci\'on del hardware NetFPGA. El mismo es utilizado en la construcción de cada nodo del prototipo; por lo que asumiendo el uso de OpenFlow en la arquitectura, es necesario partiendo del mismo obtener nodos compatibles con OpenFlow.

Existen dos estrategias bien definidas para la construcci\'on un switch OpenFlow partiendo del hardware NetFPGA y utilizando los diferentes proyectos de la plataforma. Una de ellas es programar el hardware para que se comporte como un switch compatible con el protocolo OpenFlow y la otra es programar el hardware para que se comporte como una tarjeta de red estándar, e implementar todo el comportamiento de un switch OpenFlow en software.

Para la primer estrategia se cuenta con un proyecto de la plataforma y disponible libremente en el repositorio de c\'odigo fuente. No obstante este proyecto presenta una dificultad y es que a pesar de haber sido dise\~nado para soportar en un futuro cualquier versi\'on disponible del protocolo OpenFlow, en su versi\'on actual solo soporta un conjunto reducido de funcionalidades de la versi\'on 1.0 de dicho protocolo. Como se mencion\'o anteriormente la m\'inima versi\'on de este protocolo que permite soportar el conjunto de requerimientos es la versi\'on 1.3.1. Esto conlleva a la necesidad de extender el proyecto existente, para soportar las nuevas caracter\'isticas incorporadas en las sucesivas versiones posteriores a la 1.0, \'o al menos aquellas que son esenciales para soportar las caracter\'isticas pretendidas sobre el prototipo.

Para la segunda estrategia se cuenta con un proyecto de la plataforma NetFPGA denominado ReferenceNIC. Este proyecto habilita a programar el hardware para que se comporte como una placa de red estándar. Adicionalmente se debe incluir o desarrollar herramientas que permitan implementar por software el comportamiento de un switch OpenFlow. En particular sobre este \'ultimo punto vale la pena destacar la existencia de Open vSwitch, herramienta que entre otras caracter\'isticas realiza esto mismo utilizando hardware convencional como una placa de red estándar.

En las tablas \ref{table:EOFP} y \ref{table:RNIC} se exponen comparativamente las principales ventajas y desventajas de cada alternativa.

\begin{table}[h]\centering\small
\begin{tabularx}{\textwidth}{|>{\setlength\hsize{1.0\hsize}\setlength\linewidth{\hsize}}X|>{\setlength\hsize{1.0\hsize}\setlength\linewidth{\hsize}}X|}
\hline
\multicolumn{2}{|c|}{Ventajas}\\ \hline 
\hline
Extender proyecto OpenFlow NetFPGA & ReferenceNIC + Open vSwitch\\
\hline
\begin{itemize}
\item \'Optimo aprovechamiento de la capacidad de c\'omputo y procesamiento del hardware disponible.
\item Mayor posibilidad de lograr velocidades de procesamiento competitivas con productos comerciales similares.
\item Mayor posibilidad de obtener resultados aceptables en performance para puesta en producción.

\end{itemize}
&
\begin{itemize}
\item No se tiene la necesidad de modificar o desarrollar software en el lenguaje y en el entorno de programaci\'on de la tarjeta NetFPGA.

\item Evitar desarrollar software para la NetFPGA ahorra tiempo del proyecto que se puede invertir en otras l\'ineas de trabajo igualmente importantes. 

\item Programar el hardware NetFPGA con proyectos precompilados como el ReferenceNIC requiere de licencias de software que son accesibles sin costo ya sea mediante licencias gratuitas o de prueba.
\end{itemize}
\\
\hline
\end{tabularx}
\caption[OpenFlow NetFPGA vs ReferenceNIC - Ventajas]{OpenFlow NetFPGA vs ReferenceNIC + Open vSwitch - Ventajas}
\label{table:EOFP}
\end{table}

\newpage
\begin{table}[!Ht]\centering\small
\begin{tabularx}{\textwidth}{|>{\setlength\hsize{1.0\hsize}\setlength\linewidth{\hsize}}X|>{\setlength\hsize{1.0\hsize}\setlength\linewidth{\hsize}}X|}
\hline
\multicolumn{2}{|c|}{Desventajas}\\ \hline
\hline
Extender proyecto OpenFlow NetFPGA & ReferenceNIC + Open vSwitch\\
\hline
\begin{itemize}

\item Extender el proyecto existente en s\'i mismo constituye un empresa del porte de un proyecto de fin de carrera
\item El conocimiento técnico necesario se perfila m\'as al de un Ingeniero Eléctrico que al de un Ingeniero en Computación lo cual constituye un riesgo del proyecto.
\item Desarrollar software para el hardware NetFPGA y compilarlo requiere de licencias de software costosas.
\end{itemize}

&

\begin{itemize}
\item No se aprovecha de forma óptima las capacidades de procesamiento del hardware disponible. En otras palabras se tiene hardware ``caro'' y potente en forma ociosa.
\item Los resultados obtenidos en relaci\'on al rendimiento del prototipo probablemente no sean los esperados para un equipo de producción.
\end{itemize}
\\
\hline
\end{tabularx}
\caption[OpenFlow NetFPGA vs ReferenceNIC - Desventajas]{OpenFlow NetFPGA vs ReferenceNIC + Open vSwitch - Desventajas}
\label{table:RNIC}
\end{table}

Teniendo presente el alcance del proyecto y el tiempo disponible para su ejecuci\'on, se opt\'o por la  estrategia de programar el hardware con el proyecto ReferenceNIC e implementar en software funcionalidades de OpenFlow. Con esta estrategia se logra obtener en forma rápida un prototipo de switch OpenFlow con el cual trabajar en la implementaci\'on del plano de control, desarrollando estrategias para construir servicios en una red h\'ibrida IP/MPLS, as\'i como dise\~nar casos de uso y un laboratorio de pruebas que permitan validar tecnol\'ogicamente la soluci\'on propuesta.


\section[¿Qu\'e implementaci\'on de Controlador OpenFlow utilizar?]{¿Qu\'e implementaci\'on de Controlador OpenFlow \\ utilizar?}
En el desarrollo de este proyecto, basándose en el estudio del estado del arte de las redes definidas por software realizado y en particular en el estudio sobre los controladores OpenFlow disponibles, se plantean dos alternativas que se ajustan bien a los requerimientos de este proyecto. Estas alternativas surgen de analizar todas las propuestas estudiadas y comparadas en la tabla \ref{table:Controladores}. 

Se elige dentro de este conjunto, las propuestas OpenDaylight y Ryu puesto que est\'an desarrolladas bajo la filosofía de software libre y de código abierto, cuentan con soporte, tienen una buena comunidad de usuarios, soportan el protocolo OpenFlow en su versi\'on 1.3.1 y son las dos propuestas m\'as utilizadas en la comunidad SDN/OpenFlow en la actualidad.

Para elegir el controlador a utilizar en el prototipo, se instalan ambas implementaciones y utilizando el entorno de emulación Mininet se desarrollan pequeñas aplicaciones para cada alternativa. Comparando los resultados obtenidos y evaluando principalmente la expresividad y facilidad para el desarrollo de aplicaciones en cada entorno se resuelve utilizar el controlador Ryu para la implementaci\'on del prototipo de acuerdo a las siguientes razones:

\begin{itemize}
\item El ambiente de desarrollo de OpenDaylight hace un uso intensivo de los recursos CPU y Memoria de un equipo, impidiendo ejecutarlo en un equipo de capacidades limitadas. A su vez la instalaci\'on del mismo es compleja y requiere de m\'odulos adicionales en gran medida.

\item Empíricamente el tiempo de adaptaci\'on al entorno de desarrollo de OpenDaylight es considerablemente mayor al de Ryu (curva de aprendizaje).  

\item El controlador Ryu est\'a completamente implementado en el lenguaje Python y as\'i lo debe ser también las aplicaciones que se ejecutan sobre el. Al ser Python un lenguaje interpretado, se gana agilidad en el proceso de desarrollo, verificaci\'on y despliegue de las aplicaciones en comparaci\'on a un lenguaje compilado como el utilizado por OpenDaylight (Java).

\item El enfoque de Ryu es m\'as minimalista y académico que el enfoque de OpenDaylight. Mientras que este \'ultimo est\'a enfocado a la industria y cuenta con un extenso conjunto de módulos para la compatibilidad de diferentes protocolos, Ryu ofrece una implementaci\'on m\'as liviana enfocada en el protocolo OpenFlow en exclusividad. 
\end{itemize}  
