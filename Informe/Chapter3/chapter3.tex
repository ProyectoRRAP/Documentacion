\chapter{An\'alisis del problema}

% **************************** Define Graphics Path **************************
\ifpdf
    \graphicspath{{Chapter3/Figs/Raster/}{Chapter3/Figs/PDF/}{Chapter3/Figs/}}
\else
    \graphicspath{{Chapter3/Figs/Vector/}{Chapter3/Figs/}}
\fi

El primer paso en el proceso de construcción del prototipo para la RAU2 es el análisis del problema planteado. Definir en función de requerimientos el alcance del prototipo, investigar que alternativas se tienen para construir un nodo del prototipo a partir del hardware NetFPGA, resolver que estrategia sera la utilizada para implementar servicios de VPNs, entre otros aspectos.\\

El presente capitulo esta destinado al planteo de estos aspectos, y los fundamentos sobre los que se basan las decisiones tomadas en relación a dichos aspectos. 

%En el abordaje de la problem\'atica planteada, surgen las primeras interrogantes acerca de la realidad a modelar, acerca de las tecnolog\'ias a utilizar, y en particular acerca del dise\~no en general de la soluci\'on. Algunas de estas interrogantes son ¿Como se implementan usualmente servicios de redes privadas virtuales(VPN)?, ¿Que arquitectura basada en SDN utilizar para el prototipo?, ¿Como constru\'ir un router opensource SDN a partir del hardware NetFPGA?, entre otras. 

%Las respuestas a estas preguntas, establecen los lineamientos generales sobre los que se acenta el dise\~no del prototipo. Por ello destinamos el presente cap\'itulo al planteo de las principales interrogantes tanto conceptuales, como tecnol\'ogicas, como de dise\~no, que impactan en la soluci\'on.


\section[Definición de requerimientos]{Definición de requerimientos}

Para identificar los requerimientos del prototipo de RAU2 implementado en este trabajo, inicialmente se trabaja en la identificación de posibles requerimientos sobre la RAU2; para luego contextualizar los resultados obtenidos en el alcance de este proyecto, los objetivos y resultados esperados, obteniéndose  así los requerimientos para el prototipo.\\

Basándose en [ponemos alguna referencia al borrador que nos paso eduardo...] se esperan las siguientes funcionalidades de la RAU2:

\begin{enumerate}
\item \textbf{Clasificación de tráfico:} Una de las necesidades y objetivos de la futura actualizaci\'on de la RAU, es la facilidad para clasificar tráfico. Alineado con las actuales necesidades, en particular se precisa al menos diferenciar las siguientes 3 categorías: (a)público, (b)académico y (c)servicios de contenido.

\item \textbf{Grandes volúmenes de datos (BIG Data):} En la RAU intervienen instituciones como el Instituto Pasteur, Centro Uruguayo de Imagenología Molecular (CUDIM) entre otros; en donde la generación e intercambio de grandes volúmenes de datos como los son los exámenes PET del Pasteur o una secuenciación de ADN del CUDIM entre otros.

\item \textbf{Escalabilidad:} Se espera alcanzar en un mediano plazo a un total de 11.000 Docentes, 7.000 Funcionarios y 140.000 Estudiantes; por lo que el prototipo para la nueva RAU2 debe ser escalable en la cantidad de usuarios de la infraestructura.

\item \textbf{Red de entrega de contenidos (Content Delivery Network):} Resulta sumamente útil tomar un enfoque de red de entrega de contenidos para el diseño de la red académica. Las organizaciones partícipes de la misma tienen y generan grandes volúmenes de información de gran interés por parte de otras organizaciones de la RAU. Una red de distribución de contenidos garantiza un mejor acceso en tiempo real a dicha información por parte de múltiples organizaciones en simultáneo. 
 
\end{enumerate}

Tomando el conjunto de funcionalidades y requerimientos anterior, acotando el alcance del prototipo al alcance posible de un proyecto de fin de carrera, se define el siguiente conjunto de requerimientos:

\newpage
\begin{table}[Ht!]\centering
\begin{tabularx}{\textwidth}{|>{\setlength\hsize{1.0\hsize}\setlength\linewidth{\hsize}}X|}
\hline
\multicolumn{1}{|c|}{Requerimientos Funcionales}\\ 
\hline
\begin{itemize}
\item Dada una organización o aplicación, el Sistema debe tener la capacidad de clasificar y distinguir el tráfico asociada a la misma de cualquier otro tipo de tráfico. Esto se conoce como clasificación de trafico.
\item Dado el t\'ipo de tr\'afico asociado a una organización, el Sistema debe poder asignar un porcentaje de los recursos disponibles de la red para el procesamiento del mismo.
\item Dado el t\'ipo de tr\'afico asociado a una organización, el Sistema debe poder establecer mas de un camino en la red para transportar dicho trafico implementando lo que se conoce como balanceo de carga.
\item Dadas dos subredes asociadas a una misma organización e interconectadas mediante el Sistema, se debe garantizar que la  numeración IP del tráfico generado por una de las subredes sea mantenida al ser entregado a la segunda subred; manteniendo de esta forma la identidad de los usuarios en ambas subredes. 
\end{itemize}\\
\hline
\end{tabularx}
\end{table}

\begin{table}[Ht!]\centering
\begin{tabularx}{\textwidth}{|>{\setlength\hsize{1.0\hsize}\setlength\linewidth{\hsize}}X|}
\hline
\multicolumn{1}{|c|}{Requerimientos no Funcionales}\\ 
\hline
\begin{itemize}

\item Open Source: En la medida que sea posible interesa utilizar herramientas y componentes libres y abiertas como software libre y de código abierto, hardware libre, etc.

\end{itemize}\\
\hline
\end{tabularx}
\end{table}

\section[¿Porque utilizar SDN?]{¿Porque utilizar SDN?}

Las bondades y fortalezas del enfoque de SDN fueron explicadas con anterioridad en el capitulo dedicado al estado del arte en dicha tem\'atica. En el mismo se deja constancia de las caracter\'isticas que plantea este enfoque; de las cuales muchas de ellas pueden explotarse en favor del desarrollo de funcionalidades y caracter\'isticas nuevas para el prototipo de la red acad\'emica.\\

Sin embargo, existen varias alternativas tecnol\'ogicas para la implementaci\'on del prototipo de la red acad\'emica ademas de SDN; por lo cual resulta interesante desarrollar los fundamentos en los que se basa la utilizaci\'on de este enfoque.\\  

SDN propone desacoplar los planos de datos y control, lo cual en si misma es una caracter\'istica fundamental del enfoque.\\ 
Si bien existen argumentos en favor de una infraestructura con un plano de control centralizado, tambi\'en existen argumentos v\'alidos en favor de un plano de control distribu\'ido; y esta disyuntiva entre plano centralizado y distribu\'ido no puede ser tomada a la ligera puesto que ambos enfoques pueden ser sumamente v\'alidos dependiendo del contexto en que se utilicen.\\

En el contexto del desarrollo del prototipo planteado, el enfoque centralizado adoptado por SDN resulta ventajoso sobre un enfoque distribu\'ido. Permite operar con una infraestructura de red heterog\'enea sin dependencias tecnol\'ogicas con las diferentes marcas comerciales de equipos.\\

SDN propone un mecanismo estandar para la manipulaci\'on de los diferentes dispositivos de red. Al estar estandarizada la interfaz de comunicaci\'on entre el plano de datos y el plano de control, por citar un ejemplo el protocolo OpenFlow; queda determinado el mecanismo de comunicaci\'on y las capacidades funcionales con cualquier dispositivo del plano de datos independientemente de sus, arquitectura, api de funcionalidades, de si sea de naturaleza abierto o cerrado o marca comercial.\\ 

Por otro lado el mecanismo estandar y transparente que propone SDN para manipular los diferentes dispositivos de red, brinda flexibilidad y agilidad para el desarrollo de nuevos protocolos, facilitando la investigaci\'on e innovaci\'on.\\

A su vez SDN permite que la investigaci\'on y el desarrollo de nuevos servicios y protocolos convivan con servicios comerciales en una misma infraestructura; permitiendo de esta forma utilizar adicionalmente la plataforma de hardware disponible para la investigaco\'on. 


\section[¿Porque utilizar NetFPGA?]{¿Porque utilizar NetFPGA?}

Se necesita para el desarrollo de este proyecto una plataforma tecnol\'ogica compatible con SDN, y en particular con el protocolo OpenFlow.\\
NetFPGA en especial brinda una plataforma de desarrollo y pruebas compatible con OpenFlow a un precio significativamente menor que el de un equipo comercial, y brinda funcionalidades y capacidades suficientes para el desarrollo del prototipo planteado en este trabajo.\\

A su vez la plataforma NetFPGA al utilizar hardware reconfigurable resulta ser sumamente versatil y flexible; permitiendo prototipar diferentes dispositivos en un mismo hardware f\'isico y al precio de una tarjeta NetFPGA. Esto otorga la facilidad de incorporar diferentes dispositivos al prototipo, y a un bajo costo.\\

Por otro lado no solo se puede hacer uso de los proyectos existentes para programar las tarjetas; si no que como se vera m\'as adelante en este trabajo gracias a la extensa documentaci\'on disponible sobre la plataforma y a que los proyectos existentes son de software libre y abierto resulta posible el desarrollo de nuevos proyectos con los que programar a dicho hardware. De esta forma es posible prototipar equipos con caracter\'isticas especiales pensadas en funci\'on de los requerimientos de la nueva red acad\'emica.\\

Finalmente la utilizaci\'on de hardware abierto, en conjunci\'on con software libre y de codigo abierto posibilitan la construcci\'on de lo que se dio a llamar ''router open source'' dentro del desarrollo de este prototipo de red acad\'emica. Esto tiene como ventajes una mayor versatilidad en su uso y capacidades de reutilizaci\'on puesto que puede ser mantenido, modificado y extendido por cualquier usuario o equipo de desarrollo en un futuro. Esto supone un valor agregado cosiderable. 


\section[Como se implementan VPNs)]{Como se implementan VPNs}

Existen diferentes formas de implementar sobre una infraestructura de red espec\'ifica, servicios de redes privadas virtuales o VPNs. En particular nos interesan aquellas que pueden garantizar ciertas caracter\'isticas sobre la calidad de los servicios brindados; lo que se conoce usualmente como calidad de servicios(QoS).\\

En Uruguay, el mayor proveedor de servicios en el área de las telecomunicaciones es Antel. Por lo tanto resulta útil analizar como implementa esta organización servicios de VPNs. 

Antel ofrece a sus clientes\citep{Antel} servicios de VPN Lan to Lan utilizando conexiones Ethernet Punto a Punto, servicios de VLAN Hub utilizando conexiones Ethernet, y servicios de VPN MPLS.\\

Teniendo en cuenta los requerimientos de clasificación de tr\'afico y calidad de servicios, de las alternativas mencionadas el servicio de VPN MPLS se presenta como la mejor candidata para ser imitada en el prototipo de la RAU2. Por otro lado a nivel mundial MPLS se presenta como una de las alternativas m\'as utilizadas en la implementaci\'on de redes privadas y en particular para aquellas que brinden clasificación de trafico y QoS.\\ 

En conclusión basándose en lo anterior, se escoge MPLS como alternativa tecnológica para la implementaci\'on de redes privadas en el prototipo. De aquí en adelante se toma al RFC2547\citep{rosen1999bgp} como guia para los detalles técnicos. 

\section[Arquitectura MPLS pura o IP/MPLS h\'ibrida]{Arquitectura MPLS pura o IP/MPLS h\'ibrida}

[ACA HAY QUE HABLAR DE PORQUE CONVIENE USAR IP TAMBIEN Y QUE NO SEA PURAMENTE MPLS]
\section[Elecci\'on de un algor\'itmo de ruteo]{Elecci\'on de un algor\'itmo de ruteo}

\section[Algor\'itmo para la distribusi\'on de etiquetas]{Algor\'itmo para la distribusi\'on de etiquetas}

 
\section[Que arquitectura basada en SDN utilizar]{Que arquitectura basada en SDN utilizar}

%En el cap\'itulo dedicado al estado del arte de las redes definidas por software, se explica en profundidad la arquitectura de SDN, y a partir del mismo queda completamente sanjada la diferencia entre los conceptos de SDN y OpenFlow por ejemplo.\\

%Entonces, la primer decisi\'on de dise\~no importante a tener en cuenta es por cual implementaci\'on del enfoque SDN optar, puesto que las alternativas existentes son bien diferentes.\\
 
Naturalmente, otra de las decisiones de dise\~no importantes a tomar, es la elecci\'on de una de las arquitecturas existentes basadas en el enfoque SDN.\\
  
Dentro de las alternativas existentes, OpenFlow se presenta como una opci\'on madura y probada~\citep{Ofelia}. Como se menciono en el estado del arte OpenFlow  se ha caracterizado por un desarrollo sostenido y una amplia adopci\'on tanto por la academia como por la industria, adem\'as de ser compatible con una amplia variedad de tecnolog\'ias. Debido a esto cuenta con una amplia comunidad de usuarios, extensa documentaci\'on y es soportado en varios productos comerciales\citep{Pica8}\citep{HP}\citep{Centec}\citep{SDNProductlist}. Finalmente se puede agregar que OpenFlow esta integramente desarrollado bajo la filosof\'ia opensource. 

Por estas razones, se decide por OpenFlow como la implementaci\'on del enfoque SDN a utilizar para el desarrollo del prototipo.\\

%[METER EN EL ESTADO DEL ARTE PARA JUSTIFICAR]
%nace como idea en el a\~no 2008, y luego de varias versiones de desarrollo libera su primera versi\'on oficial(OpenFlow v1.0.0) a finales del a\~no 2009. Tras ello se ha caracterizado por un desarrollo sostenido, siendo actualmente su versi\'on oficial la 1.5.
%[FIN]




%Por otro lado en las sucesivas versiones de OpenFlow, se han incorporado diferentes caracter\'isticas y funcionalidades que hacen en la actualidad a la versi\'on 1.5 de OpenFlow, una alternativa vers\'atil, flexible y completa.\\

Como todo protocolo, OpenFlow cuenta con varias versiones, en las que se incluyen conforme el protocolo va evolucionando funcionalidades nuevas. Se debe decidir entonces que versi\'on del protocolo utilizar.\\

Al momento de culminar este trabajo, la versi\'on m\'as reciente del protocolo era la 1.5, mientras que al momento de iniciarse este proyecto la versi\'on m\'as reciente era la 1.4. 

Por otro lado, la versi\'on de OpenFlow elegida debe garantizar ciertos requerimientos de diseni\~no a los que el prototipo esta sujeto. En particular interesa que garantice:

\begin{itemize}
\item Soporte para MPLS, tanto en la capacidad de reconocer los cabezales MPLS como para la manipulaci\'on de los mismos mediante las primitivas POP, PUSH y SWAP
\item Soporte para calidad de servicios(QoS)
\end{itemize}

Sobre la primer restricci\'on, OpenFlow brinda soporte completo para MPLS a partir de la versi\'on 1.3. Por otro lado en relaci\'on a calidad de servicios OpenFlow incorpora conceptos asociados a QoS desde la version [cual].\\

De esta forma, la versi\'on m\'as b\'asica de OpenFlow que asegura soporte a las restricciones de dise\~no mencionadas es la versi\'on 1.3.\\

M\'as adelante veremos en la siguiente secci\'on, la cual  esta destinada a la programaci\'on de la placa NetFPGA, que es de inter\'es trabajar con la versi\'on de OpenFlow m\'as sencilla y minimalista posible, que de soprte a todas las funcionalidades y restricciones impuestas sobre el protot\'ipo.\\ 

En conclusi\'on la versi\'on de OpenFlow a utilizar es la 1.3.

\section[Alternativas de dise\~no para el router]{Alternativas de dise\~no para el router}

Como se menciona en el capitulo 1, una de las premisas para la construcción del prototipo es la utilizaci\'on del hardware NetFPGA. Este hardware es utilizado en la construcción de cada nodo del prototipo; y asumiendo el uso de OpenFlow para la arquitectura y es necesario partiendo de este hardware obtener nodos compatibles con OpenFlow.\\ 

Existen dos estrategias bien definidas para la construcci\'on un switch OpenFlow partiendo del hardware mencionado y utilizando los diferentes proyectos de la plataforma NetFPGA. Una de ellas es programar el hardware para que se comporte como un switch compatible con el protocolo OpenFlow, y la otra alternativa es programar el hardware para que se comporte como una tarjeta de red estandar, e implementar todo el comportamiento de un switch OpenFlow en software.\\

Para la primer estrat\'egia se cuenta con un proyecto desarrollado previamente para la plataforma NetFPGA, y disponible libremente en el repositorio de software de dicho producto. No obstante este proyecto presenta una dificultad y es que a pesar de haber sido dise\~ado para soportar en un futuro cualquier versi\'on disponible del protocolo OpenFlow, en su veri\'on actual solamente soporta un conjunto reducido de funcionalidades de la versi\'on 1.0 de este protocolo. 

Como se mencion\'o anteriormente la versi\'on m\'as b\'asica de este protocolo que permite soportar el conjunto de funcionalidades relevadas como requerimientos para este proyecto es la versi\'on 1.3. Esto conlleva a la ncesidad de extender el proyecto existente, para soprtar las nuevas caracter\'isticas incorporadas en las sucesivas versiones posteriores a la 1.0; o al menos aquellas que son escenciales para soportar las caracter\'isticas pretendidas sobre el prototipo.\\

Para la segunda estrat\'egia se cuenta con un proyecto tambi\'en desarrollado previamente para la plataforma NetFPGA denominado ReferenceNIC. Este proyecto habilita a programar el hardware para que se comporte como una placa de red estandar. Adicionalmente se debe incluir o desarrollar herramientas que permitan implementar por software el comportamiento de un switch OpenFlow. En particular sobre este \'ultimo punto vale la pena destacar la existencia de Open vSwitch, herramienta que entre otras caracter\'isticas realiza esto mismo utilizando hardware convencional como una placa de red estándar.\\

A continuación se exponen comparativamente las principales ventajas y desventajas de cada alternativa.

\newpage
%%% Tabla Ventajas y Desventajas de ambas alternativas
\begin{table}[!Ht]\centering\small
\begin{tabularx}{\textwidth}{|>{\setlength\hsize{1.0\hsize}\setlength\linewidth{\hsize}}X|>{\setlength\hsize{1.0\hsize}\setlength\linewidth{\hsize}}X|}
\hline
\multicolumn{2}{|c|}{Ventajas}\\ \hline 
\hline
Extender proyecto OpenFlow NetFPGA & ReferenceNIC + Open vSwitch\\
\hline
\begin{itemize}
\item \'Optimo aprovechamiento de la capacidad de c\'omputo y procesamiento del hardware disponible.
\item Mayor posibilidad de lograr velocidades de procesamiento competitivas con productos comerciales similares.
\item Mayor posibilidad de obtener resultados aceptables, en performance y rendimiento para puesta en producción.

\end{itemize}
&
\begin{itemize}
\item No se tiene la necesidad de modificar o desarrollar software en el lenguaje y en el entorno de programaci\'on de la tarjeta NetFPGA.

\item Evitar desarrollar software para la NetFPGA ahorra tiempo de proyecto que se puede invertir en otras l\'ineas de trabajo, igualmente importantes. 

\item Programar el harware NetFPGA con proyectos precompilados como el ReferenceNIC requiere \'unicamente de licencias de software que son accesibles sin costo ya sea mediante licencias gratuitas o de prueba.
\end{itemize}
\\
\hline
\end{tabularx}
\caption[OpenFlow NetFPGA vs ReferenceNIC - Ventajas]{OpenFlow NetFPGA vs ReferenceNIC + Open vSwitch - Ventajas}
\end{table}

\begin{table}[!HT]\centering\small
\begin{tabularx}{\textwidth}{|>{\setlength\hsize{1.0\hsize}\setlength\linewidth{\hsize}}X|>{\setlength\hsize{1.0\hsize}\setlength\linewidth{\hsize}}X|}
\hline
\multicolumn{2}{|c|}{Desventajas}\\ \hline
\hline
Extender proyecto OpenFlow NetFPGA & ReferenceNIC + Open vSwitch\\
\hline
\begin{itemize}

\item Extender el proyecto existente, en si mismo constituye un empresa del porte de un proyecto de fin de carrera
\item El conocimiento técnico necesario se perfila m\'as al de un Ingeniero Eléctrico que al de un Ingeniero en Computación, lo cual constituye un riesgo del proyecto.
\item Desarrollar software para el hardware NetFPGA y compilarlo requiere de licencias de software costosas.
\end{itemize}

&

\begin{itemize}
\item No se aprovecha de forma óptima las capacidades de procesamiento del hardware disponible. En otras palabras se tiene hardwre ``caro'' y potente en forma ociosa.
\item Los resultados obtenidos en relaci\'on al rendimiento del prototipo, muy probablemente no sean los esperados para una equipo de producción.
\end{itemize}
\\
\hline
\end{tabularx}
\caption[OpenFlow NetFPGA vs ReferenceNIC - Desventajas]{OpenFlow NetFPGA vs ReferenceNIC + Open vSwitch - Desventajas}
\end{table}

\clearpage
\newpage
Teniendo presente el alcance del proyecto y el tiempo disponible para su ejecuci\'on, se opt\'o por la segunda estrat\'egia presentada. Con la segunda estrategia se logra obtener en forma rápida un prototipo de switch OpenFlow con el cual trabajar en la programaci\'on del plano de datos mediante un controlador, desarrollar estrategias para constru\'ir servicios en una red h\'ibrida IP/MPLS, as\'i como dise\~nar pruebas y un laboratorio de pruebas para validar tecnol\'ogicamente la soluci\'on propuesta.\\ 


\section[Alternativas de dise\~nio]{Plano de Control centralizado vs distribu\'ido}

%\section[Dise\~no general del prototipo]{Dise\~no general del prototipo}



