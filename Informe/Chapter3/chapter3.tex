\chapter{Dise\~no de la soluci\'on}

% **************************** Define Graphics Path **************************
\ifpdf
    \graphicspath{{Chapter3/Figs/Raster/}{Chapter3/Figs/PDF/}{Chapter3/Figs/}}
\else
    \graphicspath{{Chapter3/Figs/Vector/}{Chapter3/Figs/}}
\fi

En este cap\'itulo se exponen las principales caracter\'isticas relacionadas al dise\~no de la soluci\'on alcanzada, as\'i como la f. A su vez se exponen, de los problemas y decisiones de dise\~no encontrados aquellos que a nuestro entender resultan m\'as interesantes de mencionar en este cap\'itulo. 
 
\section[Enfoque utilizado]{Enfoque utilizado}

En el cap\'itulo dedicado al estado del arte de las redes definidas por software, se explica en profundidad la arquitectura de SDN, y a partir del mismo queda completamente sanjada la diferencia entre los conceptos de SDN y OpenFlow por ejemplo.\\

Entonces, la primer decisi\'on de dise\~no importante a tener en cuenta es por cual implementaci\'on del enfoque SDN optar, puesto que las alternativas existentes son bien diferentes.\\
 
La respuesta a esta pregunta es sencillamente OpenFlow en su versi\'on 1.3; y el porque lo explicamos a continuaci\'on.

Dentro de las arquitecturas e implementaciones que se ajustan al enfoque propuesto por SDN, como se menciona en el estado del arte existen varias y una de ellas, y quiz\'as la que m\'as ha tracendido es OpenFlow.\\
OpenFlow cuenta con un nivel de desarrollo bastante bueno, siendo la \'ultima versi\'on comercial liberada al momento de culminar este trabajo la versi\'on 1.5.
A su vez, cuenta con un extensa comunidad de usuarios y es utilizado en varias soluciones comerciales y productos de software relacionados entre los cuales podemos destacar Open vSwitch entre otros. Estos han contribuido a generar una gran cantidad de documentaci\'on y material accesible en la web, los cuales entre otras razones posicionan a OpenFlow como una de las mejores alternativas para el dise\~no del protot\'ipo. En particular [nombrar algunas soluciones tecnologicas que usarn OpenFlow].
Finalmente se puede agregar que OpenFlow esta integramente desarrollado bajo la filosof\'ia Open Source.\\

Por otro lado en las sucesivas versiones de OpenFlow, se han incorporado diferentes caracter\'isticas y funcionalidades que hacen en la actualidad a la versi\'on 1.5 de OpenFlow, una alternativa vers\'atil, flexible y completa.\\

Tomada la decisi\'on de utilizar OpenFlow para la implementaci\'on del prototipo, resta decidir en relaci\'on a esto que versi\'on utilizar del mismo.\\
Como mencionamos anteriormente al momento de culminar el desarrollo de este trabajo, la versi\'on m\'as reciente de OpenFlow era la 1.5; mientras que al momento de iniciado el desarrollo de este proyecto la versi\'on m\'as reciente era la 1.4.\\

Por otro lado dentro de las versiones disponibles de OpenFlow, se tiene que optar por aquella que cumpla con ciertas restricciones a las que el diseni\~no del prototipo esta sujeto; las m\'as importantes y que impactan en esta decis\i'on son:
\begin{itemize}
\item Soporte para MPLS, tanto en la capacidad de reconocer los cabezales MPLS como para la manipulaci\'on de los mismos mediante las primitivas POP, PUSH y SWAP
\item Soporte para tags de VLAN (esta ultima no se si ponerla)
\end{itemize}
Sobre la primer restricci\'on, OpenFlow brinda soporte completo para MPLS a partir de la versi\'on 1.3. Por otro lado en relaci\'on a la segunda restricci\'on [COMPLETAR].\\
Por ello, la versi\'on m\'as b\'asica de OpenFlow que da soporte a las restricciones de dise\~no mencionadas es la versi\'on 1.3.\\

M\'as adelante veremos en la siguiente secci\'on, que esta destinada a la programaci\'on de la placa NetFPGA, que es de inter\'es trabajar con la versi\'on de OpenFlow m\'as sencilla y minimalista que soprte todas las funcionalidades y restricciones impuestas sobre el protot\'ipo. Entonces la versi\'on de OpenFlow a utilizar es la 1.3.

\section[Alternativas de dise\~nio para el router]{Alternativas de dise\~nio para el router}

Otra conclusi\'on relevante que se desprende del estudio del estado del arte en las tarjetas NetFPGA, es la existencia de una gran variedad de proyectos open source accesibles sin costo a trav\'es del repositorio de software de NetFPGA. Dentro de este conjunto de proyectos se encuentran en particular, un proyecto para programar la tarjeta para que se comporte como un switch OpenFlow, un proyecto para programar la tarjeta para que se comporte como una tarjeta de red de un PC convencional, entre otros.\\



Otra conclusión importante, fruto del estudio del estado del arte es la existencia de
proyectos para programar las placas NetFPGA, el nivel de dicultad para programar
las placas con los mismos, y el estado de desarrollo de estos; que hacen realmente y
que no hacen.
Dada la respuesta a la pregunta anterior surge la necesidad de implementar un
switch OpenFlow. Decidimos implementar una arquitectura OpenFlow basada en
SDN, y por tanto nuestros dispositivos de la capa de Infraestructura deben ser Open-
Flow compatibles, o dicho de otro modo switches OpenFlow. Esto en el contexto de
nuestro proyecto signica que debemos hacer que las placas NetFPGA se conviertan
en un switch OpenFlow, o utilizar a las mismas como una componente más en la
construcción de un switch OpenFlow.

Aca hablamos en que existen muchos proyectos para programar las placas, y que uno de ellos es el openflow.
necesitamos tener dispositivos de openflow asi que queremos que la programacion de la placa responda a esto.


\section[Alternativas de dise\~nio]{Plano de Control centralizado vs distribu\'ido}

\section[RAUFlow]{RAUFlow}

\subsubsection{First subsub section in the third subsection}
\dots and some more in the first subsub section otherwise it all looks the same
doesn't it? well we can add some text to it and some more and some more and
some more and some more and some more and some more and some more \dots

\subsubsection{Second subsub section in the third subsection}
\dots and some more in the first subsub section otherwise it all looks the same
doesn't it? well we can add some text to it \dots

\section{Second section of the third chapter}
and here I write more \dots

\section{The layout of formal tables}
This section has been modified from ``Publication quality tables in \LaTeX*''
 by Simon Fear.

The layout of a table has been established over centuries of experience and 
should only be altered in extraordinary circumstances. 

When formatting a table, remember two simple guidelines at all times:

\begin{enumerate}
  \item Never, ever use vertical rules (lines).
  \item Never use double rules.
\end{enumerate}

These guidelines may seem extreme but I have
never found a good argument in favour of breaking them. For
example, if you feel that the information in the left half of
a table is so different from that on the right that it needs
to be separated by a vertical line, then you should use two
tables instead. Not everyone follows the second guideline:

There are three further guidelines worth mentioning here as they
are generally not known outside the circle of professional
typesetters and subeditors:

\begin{enumerate}\setcounter{enumi}{2}
  \item Put the units in the column heading (not in the body of
          the table).
  \item Always precede a decimal point by a digit; thus 0.1
      {\em not} just .1.
  \item Do not use `ditto' signs or any other such convention to
      repeat a previous value. In many circumstances a blank
      will serve just as well. If it won't, then repeat the value.
\end{enumerate}

A frequently seen mistake is to use `\textbackslash begin\{center\}' \dots `\textbackslash end\{center\}' inside a figure or table environment. This center environment can cause additional vertical space. If you want to avoid that just use `\textbackslash centering'


\begin{table}
\caption{A badly formatted table}
\centering
\label{table:bad_table}
\begin{tabular}{|l|c|c|c|c|}
\hline 
& \multicolumn{2}{c}{Species I} & \multicolumn{2}{c|}{Species II} \\ 
\hline
Dental measurement  & mean & SD  & mean & SD  \\ \hline 
\hline
I1MD & 6.23 & 0.91 & 5.2  & 0.7  \\
\hline 
I1LL & 7.48 & 0.56 & 8.7  & 0.71 \\
\hline 
I2MD & 3.99 & 0.63 & 4.22 & 0.54 \\
\hline 
I2LL & 6.81 & 0.02 & 6.66 & 0.01 \\
\hline 
CMD & 13.47 & 0.09 & 10.55 & 0.05 \\
\hline 
CBL & 11.88 & 0.05 & 13.11 & 0.04\\ 
\hline 
\end{tabular}
\end{table}

\begin{table}
\caption{A nice looking table}
\centering
\label{table:nice_table}
\begin{tabular}{l c c c c}
\hline 
\multirow{2}{*}{Dental measurement} & \multicolumn{2}{c}{Species I} & \multicolumn{2}{c}{Species II} \\ 
\cline{2-5}
  & mean & SD  & mean & SD  \\ 
\hline
I1MD & 6.23 & 0.91 & 5.2  & 0.7  \\

I1LL & 7.48 & 0.56 & 8.7  & 0.71 \\

I2MD & 3.99 & 0.63 & 4.22 & 0.54 \\

I2LL & 6.81 & 0.02 & 6.66 & 0.01 \\

CMD & 13.47 & 0.09 & 10.55 & 0.05 \\

CBL & 11.88 & 0.05 & 13.11 & 0.04\\ 
\hline 
\end{tabular}
\end{table}


\begin{table}
\caption{Even better looking table using booktabs}
\centering
\label{table:good_table}
\begin{tabular}{l c c c c}
\toprule
\multirow{2}{*}{Dental measurement} & \multicolumn{2}{c}{Species I} & \multicolumn{2}{c}{Species II} \\ 
\cmidrule{2-5}
  & mean & SD  & mean & SD  \\ 
\midrule
I1MD & 6.23 & 0.91 & 5.2  & 0.7  \\

I1LL & 7.48 & 0.56 & 8.7  & 0.71 \\

I2MD & 3.99 & 0.63 & 4.22 & 0.54 \\

I2LL & 6.81 & 0.02 & 6.66 & 0.01 \\

CMD & 13.47 & 0.09 & 10.55 & 0.05 \\

CBL & 11.88 & 0.05 & 13.11 & 0.04\\ 
\bottomrule
\end{tabular}
\end{table}
