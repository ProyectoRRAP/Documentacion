\chapter{Dise\~no de la soluci\'on}

% **************************** Define Graphics Path **************************
\ifpdf
    \graphicspath{{Chapter3/Figs/Raster/}{Chapter3/Figs/PDF/}{Chapter3/Figs/}}
\else
    \graphicspath{{Chapter3/Figs/Vector/}{Chapter3/Figs/}}
\fi

En este cap\'itulo se exponen las principales caracter\'isticas relacionadas al dise\~no de la soluci\'on alcanzada. A su vez se exponen, de los problemas y decisiones de dise\~no encontrados aquellos que a nuestro entender resultan m\'as interesantes de mencionar en este cap\'itulo. 
 
\section[Enfoque utilizado]{Enfoque utilizado}

En el cap\'itulo dedicado al estado del arte de las redes definidas por software, se explica en profundidad la arquitectura de SDN, y a partir del mismo queda completamente sanjada la diferencia entre los conceptos de SDN y OpenFlow por ejemplo.\\

Entonces, la primer decisi\'on de dise\~no importante a tener en cuenta es por cual implementaci\'on del enfoque SDN optar, puesto que las alternativas existentes son bien diferentes.\\
 
La respuesta a esta pregunta es sencillamente OpenFlow en su versi\'on 1.3; y el porque lo explicamos a continuaci\'on.

Dentro de las arquitecturas e implementaciones que se ajustan al enfoque propuesto por SDN, como se menciona en el estado del arte existen varias y una de ellas, y quiz\'as la que m\'as ha tracendido es OpenFlow.\\
OpenFlow cuenta con un nivel de desarrollo bastante bueno, siendo la \'ultima versi\'on comercial liberada al momento de culminar este trabajo la versi\'on 1.5.
A su vez, cuenta con un extensa comunidad de usuarios y es utilizado en varias soluciones comerciales y productos de software relacionados entre los cuales podemos destacar Open vSwitch entre otros. Estos han contribuido a generar una gran cantidad de documentaci\'on y material accesible en la web, los cuales entre otras razones posicionan a OpenFlow como una de las mejores alternativas para el dise\~no del protot\'ipo. En particular [nombrar algunas soluciones tecnologicas que usarn OpenFlow].
Finalmente se puede agregar que OpenFlow esta integramente desarrollado bajo la filosof\'ia Open Source.\\

Por otro lado en las sucesivas versiones de OpenFlow, se han incorporado diferentes caracter\'isticas y funcionalidades que hacen en la actualidad a la versi\'on 1.5 de OpenFlow, una alternativa vers\'atil, flexible y completa.\\

Tomada la decisi\'on de utilizar OpenFlow para la implementaci\'on del prototipo, resta decidir en relaci\'on a esto que versi\'on utilizar del mismo.\\
Como mencionamos anteriormente al momento de culminar el desarrollo de este trabajo, la versi\'on m\'as reciente de OpenFlow era la 1.5; mientras que al momento de iniciado el desarrollo de este proyecto la versi\'on m\'as reciente era la 1.4.\\

Por otro lado dentro de las versiones disponibles de OpenFlow, se tiene que optar por aquella que cumpla con ciertas restricciones a las que el diseni\~no del prototipo esta sujeto; las m\'as importantes y que impactan en esta decis\i'on son:
\begin{itemize}
\item Soporte para MPLS, tanto en la capacidad de reconocer los cabezales MPLS como para la manipulaci\'on de los mismos mediante las primitivas POP, PUSH y SWAP
\item Soporte para tags de VLAN (esta ultima no se si ponerla)
\end{itemize}
Sobre la primer restricci\'on, OpenFlow brinda soporte completo para MPLS a partir de la versi\'on 1.3. Por otro lado en relaci\'on a la segunda restricci\'on [COMPLETAR].\\
Por ello, la versi\'on m\'as b\'asica de OpenFlow que da soporte a las restricciones de dise\~no mencionadas es la versi\'on 1.3.\\

M\'as adelante veremos en la siguiente secci\'on, que esta destinada a la programaci\'on de la placa NetFPGA, que es de inter\'es trabajar con la versi\'on de OpenFlow m\'as sencilla y minimalista que soprte todas las funcionalidades y restricciones impuestas sobre el protot\'ipo. Entonces la versi\'on de OpenFlow a utilizar es la 1.3.

\section[Alternativas de dise\~no para el router]{Alternativas de dise\~no para el router}

Como se menciona anteriormente en este informe, una de las premisas para la elaboraci\'on del router open source es la utilizaci\'on del hardware NetFPGA. A su vez, partiendo de este hardware es necesario llegar a un dispositivo o switch compatible con OpenFlow.\\ 

Existen dos estrategias bien definidas para la construcci\'on un switch OpenFlow, partiendo del hardware mencionado, y utilizando los diferentes proyectos disponibles para la programaci\'on del mismo. Una de ellas es programar el hardware para que se comporte como un switch compatible con el protocolo OpenFlow. La otra alternativa es programar el hardware para que se comporte como una tarjeta de red estandar, e implementar todo el comportamiento de un switch OpenFlow en software.\\

Para la primer estrat\'egia se cuenta con un proyecto desarrollado previamente para la plataforma NetFPGA, y disponible libremente en el repositorio de software de dicho producto. No obstante este proyecto presenta una dificultad y es que a pesar de haber sido dise\~ado para soportar en un futuro cualquier versi\'on disponible del protocolo OpenFlow, en su veri\'on actual solamente soporta un conjunto reducido de funcionalidades de la versi\'on 1.0 de este protocolo. \\
Como se mencion\'o anteriormente la versi\'on m\'as b\'asica de este protocolo que permite soportar el conjunto de funcionalidades relevadas como requerimientos para este proyecto es la versi\'on 1.3. Esto conlleva a la ncesidad de extender el proyecto existente, para soprtar las nuevas caracter\'isticas incorporadas en las sucesivas versiones posteriores a la 1.0; o al menos aquellas que son escenciales para soportar las caracter\'isticas pretendidas sobre el prototipo.\\

Para la segunda estrat\'egia se cuenta con un proyecto tambi\'en desarrollado previamente para la plataforma NetFPGA, llamado ReferenceNIC. Este proyecto habilita a programar el hardware para que se comporte como una placa de red estandar. Adicionalmente se debe incluir o desarrollar herramientas que permitan implementar por software el comportamiento de un switch OpenFlow. En particular sobre este \'ultimo punto vale la pena destacar la existencia de Open vSwitch, herramienta que entre otras caracter\'isticas realiza esto mismo, utilizando hardware tradicional como una placa de red estandar.

Contextualizando ambas alternativas en el marco de la realizaci\'on de este proyecto de fin de carrera, ambas alternativas presentan sus ventajas y desventajas; no siendo ninguna de ellas a priori mejor que la otra. Por ello a continuaci\'on se exponen comparativamente las principales ventajas y desventajas de cada alternativa, a modo de sustentar la elecci\'on de una ellas. 

\newpage
%%% Tabla Ventajas y Desventajas de ambas alternativas
\begin{table}[!Ht]\centering\small
\begin{tabularx}{\textwidth}{|>{\setlength\hsize{1.0\hsize}\setlength\linewidth{\hsize}}X|>{\setlength\hsize{1.0\hsize}\setlength\linewidth{\hsize}}X|}
\hline
\multicolumn{2}{|c|}{Ventajas}\\ \hline 
\hline
Extender proyecto OpenFlow NetFPGA & ReferenceNIC + Open vSwitch\\
\hline
\begin{itemize}
\item \'Optimo aprovechamiento de la capacidad de c\'omputo y procesamiento del hardware disponible.
\item Mayor posibilidad de lograr velocidades de procesamiento competitivas con productos comerciales similares.
\item Mayor posibilidad de obtener resultados aceptables, en performance y rendimiento para puesta en producción.

\end{itemize}
&
\begin{itemize}
\item No se tiene la necesidad de modificar o desarrollar software en el lenguaje y en el entorno de programaci\'on de la tarjeta NetFPGA.

\item Evitar desarrollar software para la NetFPGA ahorra tiempo de proyecto que se puede invertir en otras l\'ineas de trabajo, igualmente importantes. 

\item Programar el harware NetFPGA con proyectos precompilados como el ReferenceNIC requiere \'unicamente de licencias de software que son accesibles sin costo ya sea mediante licencias gratuitas o de prueba.
\end{itemize}
\\
\hline
\end{tabularx}
\end{table}

\begin{table}[!HT]\centering\small
\begin{tabularx}{\textwidth}{|>{\setlength\hsize{1.0\hsize}\setlength\linewidth{\hsize}}X|>{\setlength\hsize{1.0\hsize}\setlength\linewidth{\hsize}}X|}
\hline
\multicolumn{2}{|c|}{Desventajas}\\ \hline
\hline
Extender proyecto OpenFlow NetFPGA & ReferenceNIC + Open vSwitch\\
\hline
\begin{itemize}

\item Extender el proyecto existente, en si mismo constituye un empresa del porte de un proyecto de fin de carrera
\item El conocimiento técnico necesario se perfila m\'as al de un Ingeniero Eléctrico que al de un Ingeniero en Computación, lo cual constituye un riesgo del proyecto.
\item Desarrollar software para el hardware NetFPGA y compilarlo requiere de licencias de software costosas.
\end{itemize}

&

\begin{itemize}
\item No se aprovecha de forma óptima las capacidades de procesamiento del hardware disponible. En otras palabras se tiene hardwre ``caro'' y potente en forma ociosa.
\item Los resultados obtenidos en relaci\'on al rendimiento del prototipo, muy probablemente no sean los esperados para una equipo de producción.
\end{itemize}
\\
\hline
\end{tabularx}
\end{table}


%%%%%%%%%% 

\newpage
\clearpage

\section[Alternativas de dise\~nio]{Plano de Control centralizado vs distribu\'ido}

\section[RAUFlow]{RAUFlow}

\subsubsection{First subsub section in the third subsection}
\dots and some more in the first subsub section otherwise it all looks the same
doesn't it? well we can add some text to it and some more and some more and
some more and some more and some more and some more and some more \dots

\subsubsection{Second subsub section in the third subsection}
\dots and some more in the first subsub section otherwise it all looks the same
doesn't it? well we can add some text to it \dots

\section{Second section of the third chapter}
and here I write more \dots

\section{The layout of formal tables}
This section has been modified from ``Publication quality tables in \LaTeX*''
 by Simon Fear.

The layout of a table has been established over centuries of experience and 
should only be altered in extraordinary circumstances. 

When formatting a table, remember two simple guidelines at all times:

\begin{enumerate}
  \item Never, ever use vertical rules (lines).
  \item Never use double rules.
\end{enumerate}

These guidelines may seem extreme but I have
never found a good argument in favour of breaking them. For
example, if you feel that the information in the left half of
a table is so different from that on the right that it needs
to be separated by a vertical line, then you should use two
tables instead. Not everyone follows the second guideline:

There are three further guidelines worth mentioning here as they
are generally not known outside the circle of professional
typesetters and subeditors:

\begin{enumerate}\setcounter{enumi}{2}
  \item Put the units in the column heading (not in the body of
          the table).
  \item Always precede a decimal point by a digit; thus 0.1
      {\em not} just .1.
  \item Do not use `ditto' signs or any other such convention to
      repeat a previous value. In many circumstances a blank
      will serve just as well. If it won't, then repeat the value.
\end{enumerate}

A frequently seen mistake is to use `\textbackslash begin\{center\}' \dots `\textbackslash end\{center\}' inside a figure or table environment. This center environment can cause additional vertical space. If you want to avoid that just use `\textbackslash centering'


\begin{table}
\caption{A badly formatted table}
\centering
\label{table:bad_table}
\begin{tabular}{|l|c|c|c|c|}
\hline 
& \multicolumn{2}{c}{Species I} & \multicolumn{2}{c|}{Species II} \\ 
\hline
Dental measurement  & mean & SD  & mean & SD  \\ \hline 
\hline
I1MD & 6.23 & 0.91 & 5.2  & 0.7  \\
\hline 
I1LL & 7.48 & 0.56 & 8.7  & 0.71 \\
\hline 
I2MD & 3.99 & 0.63 & 4.22 & 0.54 \\
\hline 
I2LL & 6.81 & 0.02 & 6.66 & 0.01 \\
\hline 
CMD & 13.47 & 0.09 & 10.55 & 0.05 \\
\hline 
CBL & 11.88 & 0.05 & 13.11 & 0.04\\ 
\hline 
\end{tabular}
\end{table}

\begin{table}
\caption{A nice looking table}
\centering
\label{table:nice_table}
\begin{tabular}{l c c c c}
\hline 
\multirow{2}{*}{Dental measurement} & \multicolumn{2}{c}{Species I} & \multicolumn{2}{c}{Species II} \\ 
\cline{2-5}
  & mean & SD  & mean & SD  \\ 
\hline
I1MD & 6.23 & 0.91 & 5.2  & 0.7  \\

I1LL & 7.48 & 0.56 & 8.7  & 0.71 \\

I2MD & 3.99 & 0.63 & 4.22 & 0.54 \\

I2LL & 6.81 & 0.02 & 6.66 & 0.01 \\

CMD & 13.47 & 0.09 & 10.55 & 0.05 \\

CBL & 11.88 & 0.05 & 13.11 & 0.04\\ 
\hline 
\end{tabular}
\end{table}


\begin{table}
\caption{Even better looking table using booktabs}
\centering
\label{table:good_table}
\begin{tabular}{l c c c c}
\toprule
\multirow{2}{*}{Dental measurement} & \multicolumn{2}{c}{Species I} & \multicolumn{2}{c}{Species II} \\ 
\cmidrule{2-5}
  & mean & SD  & mean & SD  \\ 
\midrule
I1MD & 6.23 & 0.91 & 5.2  & 0.7  \\

I1LL & 7.48 & 0.56 & 8.7  & 0.71 \\

I2MD & 3.99 & 0.63 & 4.22 & 0.54 \\

I2LL & 6.81 & 0.02 & 6.66 & 0.01 \\

CMD & 13.47 & 0.09 & 10.55 & 0.05 \\

CBL & 11.88 & 0.05 & 13.11 & 0.04\\ 
\bottomrule
\end{tabular}
\end{table}
