% ******************************* Thesis Appendix B ********************************

\chapter{Desaf\'ios t\'ecnicos encontrados}

Durante toda la ejecuci\'on del proyecto, nos enfrentamos a desaf\'ios y en particular de caracter t\'ecnico. Dentro de estos desaf\'ios podemos encontrar algunos que se destacan por su complejidad y dificultad para resolverlos; y otros que se destacan por su severidad como riesgo tecnol\'ogico dentro del proyecto. Por ello consideramos apropiado incluir un ap\'endice para hacer menci\'on a los mismos y a sus respectivas resoluciones.

\section{Problemas con SFPIs y Patchcoords}

Aca debemos hablar del problema que tuvimos con los patchcoords que en principio pensamos que eran de los SFPIs.

\section{Falta de licencias para suite de Xilinx ISE SDK}

Aca debemos hablar del problema que tuvimos con las licencias, en particular podriamos intentar recreear el problema para ver en que se mancaba. Detallar las licencias que se disponia y las que se consiguieron para que ande.
Esta bueno mencionar que el chino consigui licencias.

\section{Falta de driver para cable JTAG Xilinx}
Si mal no recuerdo tuvimos algun problema en relacion a esto. Tengo la vaga idea de que yo queria usar algo de linux que te permite correr drivers no recuerdo si viejos.... Fa capaz me estoy mareando con lo de la wireless de asus.

\section{Inconsistencias en las releases notes de Open vSwitch}
Bueno aca mencionamos el problema de OVS de como si bien en las faqs y en las release notes dice que funcionaba el push/pop y match de single label no andaba el pop. De como luego se corrigio esto y funcionaba para hasta 3 labels. De que de ahi en mas se usa la ovs-master

\section{MPLS Linux y Quagga LDP}
Bueno creo que es aca donde sacarnos las ganas de hablar de Quagga LDP y MPLS linux

Que es MPLS LINUX, las dos versiones que hay, que es quagga LDP las dos versiones que hay.

Con que se empezo, que problemas tuvimos. Poca documentacion o practicamente nada.

Se recompila kernel, cambia configuracion, se logra configuracion intermedia.

La version original de MPLS Linux, no andaba porque no soportaba openvswitch, no reconocia la placa etc.

Ademas no compilaba la version que estaba en repositorio. Tenia errores de compilacion, faltaban herramientas para instalar que no existian. Muchos pero muchos problemas tecnicos.

Se probaron kernels desde la 3.09.algo hasra la 3.11.26 creo generic, a unas versiones mas viejas. En la confuguracion del kernel no estaba disponible MPLS, Openvswitch, un asco todo.

Se logor hacer andar el MPLS linux nuevo con el LDP nuevo.

Se podian insertar cosas en las tablas de MPLS y el quagga por si solo andaba. No obstante no funcionaba que se insertaran desde el LDP cosas a las tablas MPLS. En particular se moria cuando intentaba insertar la primer entrada de MPLS asociada a la tabla FTN para una fec en particualr. Debugueamos el codigo, inspeccionamos y acorralamos el bug pero no pudimos solucionarlo. Se co nsidero una perdida de tiempo asi que se siguio.

[make menuconfig]

\section{Instalaci\'on de Sistema Operativo}
Instalamos Fedora 14 y no booteaba. Desactivamos el uefi de la bios y tampoco. Probamos con varias versiones de Fedora incluso con un dvd de la uri y descargada de internet. Probamos con dvd y usb.

Fedeora 20 era compatible con la mother pero no encontrabamos driver para el cable programador.

Logramos instalar Xilinx en window XP SP3 con los drivers del cable. Aqui se podian programar las placas para las pruebas de aceptacion.

Probamos con Ubuntu 14.04, y las placas se reconocian el driver existia pero las pruebas de aceptacion no encaraban (fallaban todas). Problemas con el DMA.

Se probo con Ubuntu 12.04 y se logro instalar Xilinx, conseguir driver, reconocer placa y las pruebas de aceptacion ok!. 