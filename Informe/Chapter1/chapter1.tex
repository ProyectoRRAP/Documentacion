%*******************************************************************************
%****************************** Second Chapter *********************************
%*******************************************************************************

\chapter{Introducci\'on}

\ifpdf
    \graphicspath{{Chapter1/Figs/Raster/}{Chapter1/Figs/PDF/}{Chapter2/Figs/}}
\else
    \graphicspath{{Chapter1/Figs/Vector/}{Chapter1/Figs/}}
\fi


\section{Motivación}

En la actualidad, sobre la red de Internet conviven aplicaciones académicas, comerciales y particulares con iguales prioridades. Este escenario no es apropiado para actividades de experimentación, investigación y estudio de nuevas herramientas a gran escala. A su vez los proveedores de servicios sobre Internet, “sobrevenden” el ancho de banda disponible, imposibilitando garantizar niveles mínimos de servicio; situación que se agrava en horas de mayor uso de Internet (horas pico). Este aspecto es crítico cuando se piensa en aplicaciones que requieren de niveles de calidad de servicio garantizados.

Por otro lado, si bien Internet está avanzando en el uso de enlaces de alta velocidad, esta tecnología es aún demasiado costosa para su comercialización masiva.
Por estas razones Internet no es la red más apropiada para su utilización en el medio académico.\\ 

Desde mediados de la década de los 90’s se vienen desarrollando en el mundo las redes académicas avanzadas de alta velocidad, con el objetivo de proveer una red que posibilite a docentes e investigadores colaborar en aplicaciones altamente demandantes de ancho de banda (educación a distancia, transferencia de grandes cantidades de información, acceso a equipos remotos, telemedicina, etc.), sin competir por este recurso con aplicaciones de naturaleza comercial.

En Estados Unidos el proyecto que lidera este desarrollo es Internet2, en Canadá el proyecto CA*net4, en Europa los proyectos GÉANT, y en Asia el proyecto APAN. Todas estas redes se encuentran a su vez conectadas entre sí, formando una gran red avanzada de alta velocidad de alcance mundial. En Latinoamérica, las redes académicas de Argentina, Brasil y Chile ya se encuentran integradas a Internet2.\\

Por otro lado en lo que puede considerarse como un suceso histórico en las redes académicas, 
con el objetivo de analizar la viabilidad de interconectar directamente la red académica paneuropea, GÉANT y sus equivalentes nacionales en América Latina, en el año 2002 surge la unión entre las Redes Nacionales de Educación e Investigación (RNEI) de Portugal y España (FCNN y RedIRIS) y DANTE en lo que se denominó CAESAR (Connecting All European and South(Latin) American Researchers). 

Tras meses de trabajo esta unión decanta en lo que se denomino “Declaración de Toledo”, en donde se reconoce la necesidad de crear una red troncal regional en América Latina que eventualmente se pueda conectar a GÉANT. Apenas dos semanas después, las redes latinoamericanas se unen bajo su propia agrupación denominada CLARA(Cooperación Latino Americana de Redes Avanzadas), y bajo esta nueva figura se reúnen en Río de Janeiro (Brasil) para avanzar en los acuerdos adoptados en el marco de la reunión de Toledo. El día 16 de julio del mismo año todas las redes involucradas en CLARA adscribieron a la ya denominada “Declaración de Toledo”.\\

En este contexto, la RAU2 tiene como objetivos a cumplir: unir las Instituciones Nacionales Académicas, Universidades (pública y privadas) y Centros de Investigación del Uruguay, promover el desarrollo de Redes Académicas y Científicas donde ellas hagan falta, planificar y desarrollar una red nacional, incentivar la colaboración con iniciativas similares y conectar RAU2 a Latinoamérica. 

\section{Definición del Problema}

El problema elegido como eje para el desarrollo de este trabajo es la 
construcción de lo que se dio en llamar RAU2, una modernización de la actual Red Académica Uruguaya (RAU), que estaría dotada de funciones de virtualización de redes que permitan una gran flexibilidad en su definición y uso. Como red universitaria pretende no solo mejorar en el ancho de banda disponible para desarrollar las tareas comunes con cualquier otra red de esta magnitud, sino que además busca virtualizar la red como mecanismo de asignar recursos independientes a cada Universidad o Facultad, y poder utilizarla al mismo tiempo como laboratorio de pruebas sin que esto interfiera con el funcionamiento normal de la red.\\

Interesa entonces trabajar en el desarrollo de un prototipo para la RAU2 utilizando como plataforma PCs con placas de red aceleradas en hardware reconfigurable, fruto de un desarrollo de la Universidad de Stanford denominado NetFPGA(Field Programmable Gate Arrays)\citep{NetFPGA}, y el enfoque de las Redes Definidas por Software.

\section{Objetivos}
Los objetivos planteados en este trabajo fueron participar en la implementación del prototipo RAU2 de Altas prestaciones, utilizando como punto de partida el enfoque de SDN y el hardware NetFPGA, y ejecutar un conjunto de pruebas que permitan validar estas tecnologías para su utilización en  la posible construcción de la RAU2.\\

A su vez dado la ausencia de experiencias similares  trabajando con estas tecnologías a nivel local en la órbita de la academia, se buscaba conocer a fondo estas tecnologías y en particular la arquitectura de SDN y los lenguajes de programación relacionados;  así como generar conocimiento y experiencias que puedan contribuir en trabajos a futuro. Por ello se hizo un especial hincapié en la generación de material exhaustivo sobre el estado del arte de las redes definidas por software, y una buena documentación de la experiencia obtenida utilizando el hardware NetFPGA.

\section{Resultados Esperados}
Dentro de los resultados esperados de este proyecto, uno de ellos fue el estado del arte en el enfoque de las Redes Definidas por Software(SDN) y el hardware reconfigurable NetFPGA. En particular interesaba sobre ambas tecnologías reunir suficiente información para evaluar el nivel de madurez y la posibilidad de utilizarlas en la construcción de un prototipo realista para la RAU2.\\
                   
Por otro lado un objetivo importante también fue la implementación de prototipos de gestión y control de red utilizando estrategias de SDN, en particular enfocados en las funcionalidades y capacidades que se pretenden de la RAU2.\\

Finalmente, otro objetivo y no menos importante fue el diseño e implementación de pruebas funcionales y de carga que permitan evaluar el potencial de las tecnologías utilizadas en la RAU2.

\section{Estructura del documento}
