%*******************************************************************************
%****************************** Second Chapter *********************************
%*******************************************************************************

\chapter{Introducci\'on}

\ifpdf
    \graphicspath{{Chapter1/Figs/Raster/}{Chapter1/Figs/PDF/}{Chapter2/Figs/}}
\else
    \graphicspath{{Chapter1/Figs/Vector/}{Chapter1/Figs/}}
\fi


\section[Motivaci\'on]{Motivaci\'on}

%Bueno basicamente tenemos dos motivaciones, la motivacion RAU, porque se quiere %actualizar la red academica.

%La motivacion de porque usar SDN y NetFPGA tal ves...

\section[Definici\'on del problema]{Definici\'on del problema}

\subsection[Enunciado]{Enunciado}


En pocas palabras, la problem\'atica identificada como motivaci\'on para el desarrollo de este trabajo es la siguiente:

\begin{quote}
\textit{Implementar un prototipo para la nueva Red Académica Uruguaya (RAU2) de
Altas Prestaciones, en cuanto a cobertura, velocidad, capacidad y programa-
bilidad. Explorando la idea de Software Defined Networking (Redes Defeanidas
por Software -SDN), y basándose en routers construidos en base a PCs con
placas de red aceleradas en hardware recongurable, fruto un desarrollo de
la universidad de Stanford denominado NetFPGA (Field Programmable Ga-
te Arrays). Se debe además elaborar una prueba de concepto consistente en
unos pocos nodos con enlaces ópticos sobre la que se ejecutará un conjun-
to exhaustivo de pruebas que permitan validar la idea de implantación de la
RAU2}
\end{quote}

A su vez, de la nueva Red Académica Uruguaya se conocen el siguiente conjunto de requisitos:

%Por otro se tienen identificados una serie de requisitos generales sobre la futura modernizaci\'on de la red aced\'emica, los cuales interesa tener en cuenta en el disenio de la soluci\'on planteada para el problema presentado. A continuaci\'on explicamos dichos requisitos:

\begin{enumerate}
\item \textbf{Clasificación de tráfico:} Una de las necesidades y objetivos de la futura actualizaci\'on de la RAU, es la facilidad para clasificar tráfico. Alineado con las actuales necesidades, en particular se precisa al menos diferenciar las siguientes 3 categorías: (a)público, (b)académico y (c)servicios de contenido.

\item \textbf{Grandes volúmenes de datos (BIG Data):} En la RAU intervienen instituciones como el Instituto Pasteur, Centro Uruguayo de Imagenología Molecular (CUDIM) entre otros; en donde la generación e intercambio de grandes volúmenes de datos como los son los exámenes PET del Pasteur o una secuenciación de ADN del CUDIM entre otros.

\item \textbf{Escalabilidad:} Se espera alcanzar en un mediano plazo a un total de 11.000 Docentes, 7.000 Funcionarios y 140.000 Estudiantes; por lo que el prototipo para la nueva RAU2 debe ser escalable en la cantidad de usuarios de la infraestructura.

\item \textbf{Red de entrega de contenidos (Content Delivery Network):} Resulta sumamente útil tomar un enfoque de red de entrega de contenidos para el diseño de la red académica. Las organizaciones partícipes de la misma tienen y generan grandes volúmenes de información de gran interés por parte de otras organizaciones de la RAU. Una red de distribución de contenidos garantiza un mejor acceso en tiempo real a dicha información por parte de múltiples organizaciones en simultáneo. 
 
\end{enumerate}

Por tano se espera que el prototipo este orientado a tales requisitos.

\newpage
\subsection[An\'alisis de requerimientos]{An\'alisis de requerimientos}
\label{section1.2.2}

A partir del enunciado del problema y del conjunto de requisitos planteados sobre la nueva versi\'on de la RAU, se infiere la siguiente especificaci\'on de requerimientos para el prototipo a desarrollar.


\begin{table}[H]\centering
\begin{tabularx}{\textwidth}{|>{\setlength\hsize{1.0\hsize}\setlength\linewidth{\hsize}}X|}
\hline
Requerimientos Funcionales\\
\hline
\begin{itemize}
\item Dada una organización o aplicación, el Sistema debe tener la capacidad de clasificar y distinguir el tráfico asociada a la misma de cualquier otro tipo de tráfico.
\item Dado un t\'ipo de tr\'afico, el Sistema debe poder asignar un porcentaje de los recursos disponibles de la red para el procesamiento del mismo.
\item Dadas dos subredes interconectadas mediante el Sistema, se debe garantizar que la  numeración IP del tráfico generado por una de las subredes sea mantenida al ser entregado a la segunda subred; manteniendo de esta forma la identidad de los usuarios en ambas subredes. 
\end{itemize}\\
\hline
\end{tabularx}
\end{table}

\begin{table}[H]\centering
\begin{tabularx}{\textwidth}{|>{\setlength\hsize{1.0\hsize}\setlength\linewidth{\hsize}}X|}
\hline
Requerimientos no Funcionales\\
\hline

\begin{itemize}

\item Open Source: En la medida que sea posible interesa utilizar herramientas y componentes libres y abiertas como software libre y de código abierto, hardware libre, etc.

\end{itemize}\\
\hline
\end{tabularx}
\end{table}


\clearpage
\section[¿Porque utilizar SDN?]{¿Porque utilizar SDN?}

Las bondades y fortalezas del enfoque de SDN fueron explicadas con anterioridad en el capitulo dedicado al estado del arte en dicha tem\'atica. En el mismo se deja constancia de las caracter\'isticas que plantea este enfoque; de las cuales muchas de ellas pueden explotarse en favor del desarrollo de funcionalidades y caracter\'isticas nuevas para el prototipo de la red acad\'emica.\\

Sin embargo, existen varias alternativas tecnol\'ogicas para la implementaci\'on del prototipo de la red acad\'emica ademas de SDN; por lo cual resulta interesante desarrollar los fundamentos en los que se basa la utilizaci\'on de este enfoque.\\  

SDN propone desacoplar los planos de datos y control, lo cual en si misma es una caracter\'istica fundamental del enfoque.\\ 
Si bien existen argumentos en favor de una infraestructura con un plano de control centralizado, tambi\'en existen argumentos v\'alidos en favor de un plano de control distribu\'ido; y esta disyuntiva entre plano centralizado y distribu\'ido no puede ser tomada a la ligera puesto que ambos enfoques pueden ser sumamente v\'alidos dependiendo del contexto en que se utilicen.\\

En el contexto del desarrollo del prototipo planteado, el enfoque centralizado adoptado por SDN resulta ventajoso sobre un enfoque distribu\'ido. Permite operar con una infraestructura de red heterog\'enea sin dependencias tecnol\'ogicas con las diferentes marcas comerciales de equipos.\\

SDN propone un mecanismo estandar para la manipulaci\'on de los diferentes dispositivos de red. Al estar estandarizada la interfaz de comunicaci\'on entre el plano de datos y el plano de control, por citar un ejemplo el protocolo OpenFlow; queda determinado el mecanismo de comunicaci\'on y las capacidades funcionales con cualquier dispositivo del plano de datos independientemente de sus, arquitectura, api de funcionalidades, de si sea de naturaleza abierto o cerrado o marca comercial.\\ 

Por otro lado el mecanismo estandar y transparente que propone SDN para manipular los diferentes dispositivos de red, brinda flexibilidad y agilidad para el desarrollo de nuevos protocolos, facilitando la investigaci\'on e innovaci\'on.\\

A su vez SDN permite que la investigaci\'on y el desarrollo de nuevos servicios y protocolos convivan con servicios comerciales en una misma infraestructura; permitiendo de esta forma utilizar adicionalmente la plataforma de hardware disponible para la investigaco\'on. 


\section[¿Porque utilizar NetFPGA?]{¿Porque utilizar NetFPGA?}

Se necesita para el desarrollo de este proyecto una plataforma tecnol\'ogica compatible con SDN, y en particular con el protocolo OpenFlow.\\
NetFPGA en especial brinda una plataforma de desarrollo y pruebas compatible con OpenFlow a un precio significativamente menor que el de un equipo comercial, y brinda funcionalidades y capacidades suficientes para el desarrollo del prototipo planteado en este trabajo.\\

A su vez la plataforma NetFPGA al utilizar hardware reconfigurable resulta ser sumamente versatil y flexible; permitiendo prototipar diferentes dispositivos en un mismo hardware f\'isico y al precio de una tarjeta NetFPGA. Esto otorga la facilidad de incorporar diferentes dispositivos al prototipo, y a un bajo costo.\\

Por otro lado no solo se puede hacer uso de los proyectos existentes para programar las tarjetas; si no que como se vera m\'as adelante en este trabajo gracias a la extensa documentaci\'on disponible sobre la plataforma y a que los proyectos existentes son de software libre y abierto resulta posible el desarrollo de nuevos proyectos con los que programar a dicho hardware. De esta forma es posible prototipar equipos con caracter\'isticas especiales pensadas en funci\'on de los requerimientos de la nueva red acad\'emica.\\

Finalmente la utilizaci\'on de hardware abierto, en conjunci\'on con software libre y de codigo abierto posibilitan la construcci\'on de lo que se dio a llamar ''router open source'' dentro del desarrollo de este prototipo de red acad\'emica. Esto tiene como ventajes una mayor versatilidad en su uso y capacidades de reutilizaci\'on puesto que puede ser mantenido, modificado y extendido por cualquier usuario o equipo de desarrollo en un futuro. Esto supone un valor agregado cosiderable. 