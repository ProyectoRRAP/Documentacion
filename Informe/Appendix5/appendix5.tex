% ******************************* Thesis Appendix C ********************************

\chapter{Ejecuci\'on de RAUF}
\label{appendix5}

% **************************** Define Graphics Path **************************
\ifpdf
    \graphicspath{{Appendix5/Figs/Raster/}{Appendix5/Figs/PDF/}{Appendix5/Figs/}}
\else
    \graphicspath{{Appendix5/Figs/Vector/}{Appendix5/Figs/}}
\fi

En el presente anexo se incluyen una peque\~na guía ilustrativa para la ejecuci\'on de la aplicaci\'on RAUFlow, así como una pequeña descripción de la interfaz web.\\

Para ejecutar la aplicaci\'on es necesario ejecutar iniciar el controlador Ryu y ejecutar las cuatro aplicaciones Ryu incluidas en la arquitectura. Para ello se ejecuta el siguiente comando en el directorio Proyecto/ryu-master dentro de la estructura de directorios del proyecto.

\begin{center}
./bin/ryu run --observe-links ryu/app/proyecto/businessLogic/RAUFlowApp.py

\end{center}

\begin{figure}[h] 
\centering    
\includegraphics[width=1.0\textwidth]{Snap10}
\caption[Ejecuci\'on RAUFlow]{Ejecuci\'on RAUFlow}
\label{fig:Img2}
\end{figure}

Este comando levanta el controlador Ryu y le pasa como parametro la aplicaci\'on RAUFlowApp la cual internamente instancia las otras tres aplicaciones Ryu.

Por otro lado la aplicaci\'on instancia la API REST de servicios, por defecto en el puerto 8080 en la direcci\'on localhost. Ambos valores pueden configurarse desde el archivo wsgi.py.


\begin{figure}[h] 
\centering    
\includegraphics[width=1.2\textwidth]{SnapApp}
\caption[Interfaz RAUFlow capturas]{Interfaz RAUFlow capturas}
\label{fig:Img2}
\end{figure}