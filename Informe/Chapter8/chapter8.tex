\chapter{Conclusiones}

% **************************** Define Graphics Path **************************
\ifpdf
    \graphicspath{{Chapter8/Figs/Raster/}{Chapter8/Figs/PDF/}{Chapter8/Figs/}}
\else
    \graphicspath{{Chapter8/Figs/Vector/}{Chapter8/Figs/}}
\fi

En este cap\'itulo se resumen los principales resultados, logros y conclusiones de este trabajo. Luego se enumeran las principales l\'ineas de trabajo a futuro identificadas, entre las cuales se incluyen mejoras y extensiones al prototipo, como posibles l\'ineas de investigaci\'on.

\section{Conclusiones}
Se realiz\'o una investigaci\'on en profundidad del estado del arte de las redes definidas por software  
 (SDN) y la plataforma de hardware NetPGA, presentando un res\'umen de los resultados obtenidos en el cap\'itulo 2 de este trabajo.

Por otro lado se logr\'o desarrollar un prototipo funcional utilizando el hardware NetFPGA y el enfoque de SDN, dotado de funcionalidades para la creaci\'on y gesti\'on de servicios de redes privadas virtuales. Este prototipo es la prueba de que es posible implementar una red IP/MPLS en la que la inteligencia de los dispositivos de red convencionales es extra\'ida y colocada en una entidad centralizada.

El prototipo alcanzado se compone de un dispositivo de red denominado RAU-Switch y una aplicaci\'on de gesti\'on denominada RAUFlow. Orientado a una futura reproducci\'on de este trabajo, se gener\'o un manual de construcci\'on de RAU-Switch donde se detallan todas las componentes de hardware y software utilizadas, junto con el procedimiento de instalaci\'on y configuraci\'on de las mismas, para obtener un dispositivo id\'entico al desarrollado en este trabajo. Adem\'as se cre\'o un sitio del proyecto en la plataforma Github (Proyecto RRAP\cite{GitRRAP}) en donde se comparti\'o con la comunidad el c\'odigo fuente de todas las componentes de software desarrolladas en el marco de este proyecto, el conocimiento y experiencia generados en relaci\'on a las herramientas utilizadas. 

Por otro lado se diseñ\'o e implement\'o un laboratorio de experimentaci\'on sobre el cual se ejecutaron una serie de pruebas orientadas a la validaci\'on funcional del prototipo referidas a dos casos de uso representativos, ellos fueron la configuración de una VPN de capa 3 multipunto y una VPN de capa 2 punto a punto, sobre los cuales a su vez se ejecutaron una serie de pruebas para verificar la correcta implementaci\'on. De esta forma se logr\'o validar en lo que concierne a los requerimientos planteados, la aplicabilidad del enfoque SDN en la construcci\'on de la RAU2.

Adem\'as se gener\'o una publicaci\'on cient\'ifica en la cual se presentan los resultados obtenidos por este trabajo en el contexto de la construcci\'on de un prototipo para la RAU2 bajo el nombre "RAU2 testbed: a network prototype for evolved service experimentation" \cite{RauflowArticle}. Este art\'iculo fue presentado en la conferencia Latin American Network Operations and Management Symposium (LANOMS) y aceptado en formato de poster.

No menos importante, se logr\'o integrar el equipo de desarrollo con la comunidad de NetFPGA a trav\'es de la participaci\'on en la lista oficial de correos; contribuyendo desde la experiencia obtenida en la contestaci\'on de dudas y reportando dos errores importantes en el proyecto ReferenceNIC. A su vez se gener\'o un grupo de trabajo local integrando profesionales del SeCIU, Centro de Capacitaci\'on y Desarrollo de ANTEL y del Centro Universitario de la Regi\'on Este (CURE), en el cual se organizaron reuniones quincenales durante un per\'iodo de casi ocho meses para la puesta en com\'un y generaci\'on de experiencia y conocimiento en el \'area de SDN.

Finalmente, pese q que el dispositivo RAU-Switch diseñado est\'a fuertemente limitado en su rendimiento y performance por su implementaci\'on mayoritariamente en software y a pesar de que no se pudieron realizar pruebas comparativas con productos comerciales similares en funcionalidades, se logr\'o validar la utilizaci\'on de esta plataforma en la construcci\'on de un dispositivo compatible con OpenFlow. Adem\'as se identific\'o el camino a seguir para la construcci\'on de un prototipo con mejores prestaciones explotando al m\'aximo las capacidades del hardware disponible.  

\section{Trabajo a futuro}
En este trabajo no se compara el rendimiento del prototipo con productos comerciales similares en funcionalidades dado que estos implementan el plano de datos de OpenFlow en hardware, mientras que el prototipo lo implementa en software (Open vSwitch). De esta forma los tiempos de rendimiento ser\'ian incomparables.

Por ello interesa fuertemente extender el proyecto OpenFlow de la plataforma NetFPGA para que implemente el protocolo como mínimo en la versi\'on 1.3.1 y as\'i poder realizar una evaluaci\'on experimental que permita validar desde el punto de vista del rendimiento la aplicaci\'on de las tecnolog\'ias utilizadas en la construcci\'on de la RAU2.

El algoritmo de ruteo implementado basa su m\'etrica solamente en el costo asociado a un enlace de red. Extender la definici\'on de esta m\'etrica, contemplando otros atributos de un enlace como el ancho de banda disponible o tecnolog\'ia (fibra \'optica, enlace de cobre, etc) y a la vez contemplar restricciones como garantizar un ancho de banda m\'inimo, cantidad de enlaces atravesados, incluir \'o excluir nodos por los que pasar\'a el tr\'afico y demoras de extremo a extremo (en otras palabras implementar un CSPF) redituaría en la capacidad para incorporar funcionalidades de calidad de servicio.

De la mano de la extensi\'on del algoritmo de ruteo a un algoritmo CSPF, se puede trabajar en el desarrollo de funcionalidades avanzadas que permitan implementar t\'ecnicas de Ingenier\'ia de Tr\'afico, desarrollando as\'i un prototipo m\'as flexible en la asignaci\'on de los recursos disponibles y con mejor calidad en los servicios brindados.

Un servicio queda definido en el sistema por los nodos e interfaces de entrada y salida a la red y las caracter\'isticas del tr\'afico asociado. La incorporaci\'on de nuevas dimensiones a la definici\'on de un servicio, como dimensiones de calidad de servicios (QoS) o simplemente la dimensi\'on tiempo representar\'ia una gran mejora funcional y un salto de calidad en el aprovechamiento de las capacidades de la infraestructura del prototipo. Por ejemplo incorporando la dimensi\'on tiempo se podr\'ian definir servicios para rangos horarios, mejorando la precisi\'on con la que se distribuye el ancho de banda disponible.

En el prototipo se almacena en memoria informaci\'on que es ingresada por un usuario de la aplicaci\'on RAUFlow. En particular se almacenan datos extra de cada nodo e interfaz y se almacena la definici\'on de los servicios. Esta informaci\'on se pierde en caso de que la aplicaci\'on sea interrumpida puesto que no se persiste de forma no volátil. Resulta interesante entonces incorporar a la arquitectura de RAUFlow una capa de persistencia que permita guardar de forma no vol\'atil esta informaci\'on, adem\'as de la definici\'on de estrategias para la reconstrucci\'on de servicios cuando se carga esta informaci\'on eventualmente en una topolog\'ia de red diferente a la inicial.

Almacenar en memoria y de forma centralizada la informaci\'on topol\'ogica de red, informaci\'on asociada a servicios y eventualmente funcionalidades de QoS e Ingenier\'ia de tr\'afico, as\'i como ejecutar algoritmos que utilicen intensivamente estos datos como un CSPF centralizado puede acarrear serias barreras de escalabilidad en el prototipo. Una l\'inea de investigaci\'on interesante ser\'ia el desarrollo de una jerarqu\'ia de Controladores, cada uno responsable del plano de control de una porci\'on de la topolog\'ia global. De esta forma se generar\'ian islas SDN en donde el tr\'afico interno es resuelto por el controlador local y se consulta al Controlador global para resolver la forma en que se enruta tr\'afico entre diferentes islas. A su vez se puede crecer en la cantidad de niveles dentro de la jerarqu\'ia tanto como se quiera.

Finalmente y no menos importante se puede trabajar en el desarrollo de nuevas funcionalidades en RAUFlow, así como en la mejora de las ya existentes. Algunas de las funcionalidades que se pueden incorporar son la capacidad para definir manualmente caminos, indicando los nodos por los que se quiere  
pasar, soportar m\'ultiples caminos para un mismo par de nodos origen y destino y as\'i poder implementar balanceo de carga, creaci\'on de una VPN multipunto autom\'aticamente a partir de una lista de nodos e interfaces involucradas. 

