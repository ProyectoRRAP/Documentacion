\chapter{Laboratorio de pruebas}
\label{chapter6}

% **************************** Define Graphics Path **************************
\ifpdf
    \graphicspath{{Chapter6/Figs/Raster/}{Chapter6/Figs/PDF/}{Chapter6/Figs/}}
\else
    \graphicspath{{Chapter6/Figs/Vector/}{Chapter6/Figs/}}
\fi

Para validar funcionalmente el prototipo es preciso la construcción de un laboratorio de experimentación sobre el cual ejecutar un conjunto de pruebas e implementar algunos casos de uso representativos. Dentro de las pruebas funcionales se busca verificar aspectos importantes de la implementaci\'on como la clasificación de tr\'afico, el algoritmo de ruteo dinámico y el algoritmo de distribución de etiquetas.

Por otro lado con la implementaci\'on de algunos casos de uso representativos se busca validar la utilización del enfoque OpenFlow/SDN en la construcci\'on a futuro de la RAU2.\\

El siguiente capitulo esta destinado a la presentación del laboratorio de pruebas construido y a la descripción de las diferentes pruebas realizadas.

% **************************** Construccion del TestBed ************************** 
\section{Definición del Laboratorio}

Como se menciona anteriormente uno de los objetivos de este laboratorio de experimentaci\'on es verificar el correcto funcionamiento de la implementaci\'on realizada, en particular en los aspectos críticos de la misma como la capacidad para clasificar trafico y los diferentes algoritmos de ruteo, distribución de etiquetas, actualización topologica, etc. 

Para esto es necesario construir un prototipo con suficientes nodos y enlaces como para poder definir varios servicios de redes privadas. Tambi\'en interesa la existencia de caminos alternativos entre un par de nodos origen y destino para poder comprobar el funcionamiento del algoritmo de ruteo y evaluar el comportamiento del prototipo cuando la topolog\'ia cambia, desconectando enlaces, apagando nodos, etc.\\

Teniendo en cuanta los requerimientos mencionados se construye un laboratorio de pruebas con la siguiente topolog\'ia (ver figura \ref{fig:LaboratorioDePruebasTopo}).
  
\begin{figure}[ht!] 
\centering    
\includegraphics[width=0.85\textwidth]{Topologia}
\caption[Laboratorio de pruebas - Topolog\'ia]{Laboratorio de pruebas - Topolog\'ia}
\label{fig:LaboratorioDePruebasTopo}
\end{figure}

El laboratorio se compone de cuatro nodos implementados en base al dispositivo RAU-Switch conectados entre si con enlaces de fibra \'optica multimodo. Cada nodo esta conectado a los otros tres nodos de la topolog\'ia implementando de esta forma una topolog\'ia full mesh de cuatro nodos.

A su vez cada nodo esta conectado al controlador SDN del prototipo mediante una interfaz de red de 10Mb (interfaces \textbf{eth0}).\\

Por la forma que tiene la topolog\'ia, los cuatro nodos nombrados \textit{Galois}, \textit{Poisson}, \textit{Oz} y \textit{Alice} son nodos de borde. Esto quiere decir que RAUFlow los considera como nodos habilitados para ser nodo de ingreso \'o nodo de egreso en la definici\'on de servicios de redes privadas.

Por otro lado cada nodo cuenta con una interfaz de red de 100Mb (interfaz \textbf{eth1}) utilizada como interfaz externa para la conexi\'on con otras subredes. Estas interfaces son utilizadas por RAUFlow para la definici\'on de servicios de VPN como los puntos de entrada y salida de tr\'afico. Por tanto cada una de estas interfaces se encuentra directamente conectada a la subred de una VPN en particular.\\  

Como se menciona anteriormente en lacp\'itulo 5, la representaci\'on utilizada para modelar una topolog\'ia de red es la de un multigrafo dirigido ponderado. En la figura \ref{fig:LaboratorioDePruebasCostos} se muestra esta representaci\'on para la topolog\'ia del prototipo. Como puede apreciarse en la im\'agen cada enlace tiene su respectivo costo asociado.

Por simplicidad en el diagrama se han obviado los enlaces existentes entre el Controlador y cada uno de los nodos. Como se menciona en el cap\'itulo 4 la instancia de Quagga ejecutada en el controlador tiene como \'unico objetivo contar con un acceso local a la LSDB. Por ello conceptualmente el costo asociado a cada uno de estos enlaces es infinito, lo cual en pr\'actica se realiza asignando el valor 65535.  

\begin{figure}[ht!] 
\centering    
\includegraphics[width=0.85\textwidth]{TopologiaCostos}
\caption[Laboratorio de pruebas - Costos de la topolog\'ia]{Laboratorio de pruebas - Costos de la topolog\'ia}
\label{fig:LaboratorioDePruebasCostos}
\end{figure}

Sobre este laboratorio se implementan dos casos de uso representativos: (a) VPN de capa 3 y (b) VPN de capa 2. Sobre cada uno de estos casos de uso a su vez se ejecutan una serie de pruebas orientadas a verificar el correcto funcionamiento de las componentes m\'as importantes en la implementaci\'on de RAUFlow y RAU-Switch.

A continuaci\'on se describen los resultados obtenidos en la implementaci\'on de estos casos de uso y la ejecuci\'on de estas pruebas, comenzando por el caso de uso VPN de capa 3.

\section{VPN de capa 3}

Las redes privadas virtuales de capa 3 son un tipo de servicio comunmente brindado por un operador de red y es en particular uno de los servicios que se quiere implementar en la RAU2. En particular sobre este tipo de redes privadas puede implementarse clasificaci\'on de tr\'afico por tipo de aplicaci\'on o numeraci\'on de capa 3 por ejemplo.\\

En este trabajo se implementan dos escenarios diferentes para este tipo de red privada: (a) red privada multipunto con una \'unica organizaci\'on y tres sucursales f\'isicamente separadas, (b) red privada punto a punto con dos organizaciones, cada una de ellas con dos sucursales f\'isicamente separadas.

A continuaci\'on se detalla la construcci\'on de ambos escenarios y los resultados obtenidos.

\subsection{Escenario 1}

Este escenario representa una red privada multipunto de capa 3. En el mismo se tiene una sola organizaci\'on o red privada dividida en 3 sucursales o subredes físicamente separadas.

\begin{figure}[ht!] 
\centering    
\includegraphics[width=1.0\textwidth]{CU1P1}
\caption[VPN de capa 3 - Escenario 1]{VPN de capa 3 - Escenario 1}
\label{fig:CUP1}
\end{figure}

Sobre este escenario se ejecutan una serie de pruebas orientadas a verificar los siguientes aspectos relacionados al prototipo:

\begin{enumerate}
\item Algoritmo de ruteo
\item Algoritmo de distribución de etiquetas
\item Clasificaci\'on de tr\'afico
\item Actualizaci\'on de rutas cuando cambia la topolog\'ia
\item Capacidad para crear VPN multipunto
\end{enumerate}

Para construir la red privada multipunto uniendo las tres subredes mencionadas y adem\'as verificar los puntos anteriores, se instancian los siguientes servicios(ver tabla \ref{table:TablaFlujos}). Por cada par de subredes se crean dos servicios (un servicio para cada sentido del tr\'afico).\\

\begin{table}[h]
\begin{tabular}{| l | l | l | p{4cm} | p{4cm} |}
\hline
Nombre & Ingreso & Egreso & Clasificación & Descripción \\ \hline

\crule[Aquamarine]{0.3cm}{0.3cm} S1 & Galois - eth1 & Oz - eth1 & ip\_src=20.20.20.128/26 ip\_dst=20.20.20.0/26 eth\_type=0x0800 & Tr\'afico IP de Subred A a Subred C \\ \hline

\crule[Red]{0.3cm}{0.3cm} S2 & Oz - eth1 & Galois - eth1 & ip\_src=20.20.20.0/26 ip\_dst=20.20.20.128/26 eth\_type=0x0800 & Tr\'afico IP de Subred C a Subred A \\ \hline

\crule[ForestGreen]{0.3cm}{0.3cm} S3 & Galois - eth1 & Poisson - eth1 & ip\_src=20.20.20.128/26 ip\_dst=20.20.20.64/26 eth\_type=0x0800 & Tr\'afico IP de Subred A a Subred B \\ \hline

\crule[LimeGreen]{0.3cm}{0.3cm} S4 & Poisson - eth1 & Galois - eth1 & ip\_src=20.20.20.64/26 ip\_dst=20.20.20.128/26 eth\_type=0x0800 & Tr\'afico IP de Subred B a Subred A \\ \hline

\crule[RoyalPurple]{0.3cm}{0.3cm} S5 & Poisson - eth1 & Oz - eth1 & ip\_src=20.20.20.64/26 ip\_dst=20.20.20.0/26 eth\_type=0x0800 IP & Tr\'afico de Subred B a Subred C \\ \hline

\crule[YellowOrange]{0.3cm}{0.3cm} S6 & Oz - eth1 & Poisson - eth1 & ip\_src=20.20.20.0/26 ip\_dst=20.20.20.64/26 eth\_type=0x0800 & Tr\'afico IP de Subred C a Subred B \\ \hline 

\end{tabular}
\vspace{0.3cm}
\caption[CU1 - Escenario 1, servicios creados]{CU1 - Escenario 1, servicios creados}
\label{table:TablaFlujos}
\end{table}

Para cada uno de estos servicios adem\'as se indica el etherype 0x0800 correspondiente al tipo de tr\'afico IPv4.\\

Este conjunto de servicios establece el procesamiento de paquetes de capa 3 para las diferentes subredes de forma de conectarlas en una red privada de capa 3. Sin embargo no alcanzan para la implementaci\'on de la misma. Es necesario definir un conjunto de servicios adicionales.\\

Para fija ideas, cuando el host H1 en la sub red A envia un paquete con destino al host H3 en la subred B, primero envía el paquete al gateway de la subred A (r1) el cual entiende que el paquete debe ser reenviado al router de borde de la red del prototipo al cual esta conectada la subred, en este caso Galois. No obstante en este paso r1 no conoce la dirección de capa de enlace (direcci\'on MAC) que debe colocarse al paquete para ser reenviado a Galois; por ello utilizando el protocolo ARP (Address Resolution Protocol) inicia una consulta ARP con la direcci\'on del siguiente paso, el router r2 (192.168.2.113).

Para que esta consulta ARP pueda atravesar la red del laboratorio, así como su respuesta y para todas las combinaciones de consultas ARP en el escenario planteado, se instancian los siguientes servicios (ver tabla \ref{table:TablaFlujos2}). 

\begin{table}[h]
\begin{tabular}{| l | l | l | p{4cm} | p{4cm} |}
\hline
 
S7 & Galois - eth1 & Oz - eth1 & arp\_tpa=192.168.2.112 eth\_type=0x0806 & Consultas ARP Subred A a Subred C \\ \hline

S8 & Oz - eth1 & Galois - eth1 & arp\_tpa=192.168.2.111 eth\_type=0x0806 & Consultas ARP Subred C a Subred A \\ \hline

S9 & Galois - eth1 & Poisson - eth1 & arp\_tpa=192.168.2.113 eth\_type=0x0806 & Consultas ARP Subred A a Subred B \\ \hline

S10 & Poisson - eth1 & Galois - eth1 & arp\_tpa=192.168.2.111 eth\_type=0x0806 & Consultas ARP Subred B a Subred A \\ \hline

S11 & Poisson - eth1 & Oz - eth1 & arp\_tpa=192.168.2.112 eth\_type=0x0806 & Consultas ARP Subred B a Subred C \\ \hline

S12 & Oz - eth1 & Poisson - eth1 & arp\_tpa=192.168.2.113 eth\_type=0x0806 & Consultas ARP Subred C a Subred B \\ \hline

\end{tabular}
\vspace{0.3cm}
\caption[CU1 - Escenario 1, servicios adicionales]{CU1 - Escenario 1, servicios adicionales}
\label{table:TablaFlujos2}
\end{table}

Ahora si, se tiene definido el conjunto de servicios que implementa la red privada de capa 3 multipunto en dicho escenario, utilizando el prototipo. A continuaci\'on se detallan las pruebas realizadas para cada uno de los aspectos mencionados anteriormente.

\subsubsection{Verificaci\'on de Algoritmo de ruteo}
En esta secci\'on se eval\'ua el correcto funcionamiento del algoritmo de ruteo. Para ello se comparan los caminos computados por el algoritmo para cada servicio, con los respectivos mejores caminos te\'oricos(previamente calculados de forma manual). 

Para cada camino se verifican dos cosas: (1) que el camino calculado se corresponde con el camino te\'orico y (2) que el camino es correctamente instalado en forma de reglas de reenvío (en base a conmutaci\'on de etiquetas) en las respectivas tablas de flujos OpenFlow de cada nodo en el camino.\\

Como se ha mencionado, los caminos te\'oricos son calculados manualmente observando los costos de la topolog\'ia (figura \ref{fig:LaboratorioDePruebasCostos}). Los resultados de este procedimiento se muestran en la figura \ref{fig:CUP1Caminos}. Cada camino es identificado mediante el código de color de cada servicio, omitiendo los servicios de la tabla \ref{table:TablaFlujos2} con el objetivo de simplificar el diagrama.\\

\begin{figure}[ht!] 
\centering    
\includegraphics[width=1.0\textwidth]{CU1P1Caminos}
\caption[Escenario 1 - Caminos para servicios]{Escenario 1 - Caminos para servicios}
\label{fig:CUP1Caminos}
\end{figure}

Para obtener los caminos calculados por el algoritmo de ruteo, se analizan las tablas de flujos de Open vSwitch en cada nodo de la red del laboratorio. A partir del contenido de estas tablas fácilmente puede reconstruirse el camino previamente calculado.\\

En las figuras ~\ref{fig:CU1P1DumpFlows1}~-~\ref{fig:CU1P1DumpFlows4} se muestran las tablas de flujos asociadas a cada nodo del laboratorio. Para conseguir esta informaci\'on se utiliza el comando \textbf{dump-flows} de la herrmaienta Open vSwitch. Sin embargo puede obtenerse tambi\'en esta informaci\'on desde la propia interfaz gr\'afica de RAUFlow.\\

\newpage
\begin{figure}[ht!] 
\centering    
\includegraphics[width=1.0\textwidth]{LabE1P1Gal}
\caption[Tabla de flujos ovs - Galois]{Tabla de flujos ovs - Galois}
\label{fig:CU1P1DumpFlows1}
\end{figure}

\begin{figure}[h!] 
\centering    
\includegraphics[width=1.0\textwidth]{LabE1P1Oz}
\caption[Tabla de flujos ovs - Oz]{Tabla de flujos ovs - Oz}
\label{fig:CU1P1DumpFlows2}
\end{figure}

\begin{figure}[h!] 
\centering    
\includegraphics[width=1.0\textwidth]{LabE1P1Al}
\caption[Tabla de flujos ovs - Alice]{Tabla de flujos ovs - Alice}
\label{fig:CU1P1DumpFlows3}
\end{figure}

\begin{figure}[h!] 
\centering    
\includegraphics[width=1.0\textwidth]{LabE1P1Poi}
\caption[Tabla de flujos ovs - Poisson]{Tabla de flujos ovs - Poisson}
\label{fig:CU1P1DumpFlows4}
\end{figure}

\newpage
Asumiendose la notaci\'on $(n, i)$ para referirse a un enlace, donde \textit{n} indica nodo origen e \textit{i} interfaz de reenvío en \textit{n} para el próximo salto, entonces $<(n_1, i_1), \dots, (n_k, i_k)>$ puede usarse para denotar un camino en el laboratorio. De esta forma se pueden comparar f\'acilmente los caminos te\'oricos con los calculados.\\

Tomando como ejemplo el caso del servicio S3, el camino te\'orico puede denotarse de la siguiente forma:
 
$$<(Galois, nf_2)>$$

En otras palabras todo tr\'afico IP con origen en la subred A y destino a la subred B, es encaminado a través de la interfaz $nf_2$ en el nodo de ingreso \textit{Galois}. Luego en el nodo de egreso \textit{Poisson} es reenviado por la correspondiente interfaz del servicio.\\

Analizando las tablas de flujos de los nodos \textit{Galois} y \textit{Poisson}, puede comprobarse fácilmente la correspondencia entre el camino calculado y el camino te\'orico.

Por un lado en la tabla de flujos del nodo \textit{Galois} se tiene el siguiente flujo:

%Por un lado, acorde a la definci\'on del servicio, Galois debería reenviar todo tr\'afico de tipo \textbf{ip} que ingresa por la interfaz eth1(la cual se corresponde con el n\'umero de puerto openflow 1), con origen en la subred 20.20.20.64/26 y destino 20.20.20.0/64 por la interfaz nf1(la cual se corresponde con el n\'umero de puerto openflow 3).

\begin{figure}[h]
\textit{cookie=0.0, duration=531.354s, table=0, n\_packets=0, n\_bytes=0, priority=5, \\
ip,in\_port=1, nw\_src=20.20.20.128/26,nw\_dst=20.20.20.0/26 \\
actions=dec\_ttl,push\_mpls:0x8847,set\_field:30->mpls\_label,set\_mpls\_ttl(128),output:4}
\centering
\caption{Flujo 1}
\label{fig:Flujo1}
\end{figure}

Este flujo toma todo paquete recibido por el puerto openflow identificado con el n\'umero 1(interfaz eth1), numeración IP origen 20.20.20.128/26(subred A) y destino 20.20.20.64/26 (subred B), decrementa el TTL del paquete, coloca un cabezal mpls con la etiqueta 32 y finalmente lo reenvía por el puerto openflow identificado con el n\'umero 4(interfaz nf2). Observar que el valor de la etiqueta mpls colocado es el utilizado para identificar el servicio(etiqueta interna). \\

%Notese adem\'as la acci\'on \textbf{dec\_ttl} en el flujo. Esta acci\'on es utilizada en cada nodo de borde en la definici\'on de un servicio para decrementar el ttl de un paquete ip cada vez que ingresa a la red del prototipo (en este caso a la red del laboratorio). Observese tambi\'en el valor de la etiqueta mpls(34) colocado en el paquete para identificar el servicio(etiqueta interna).

El procesamiento de los paquetes asociados al servicio S3 en el nodo \textit{Poisson}, esta dado por el siguiente flujo:

\begin{figure}[h]
\textit{cookie=0.0, duration=655.076s, table=0, n\_packets=0, n\_bytes=0, priority=5, \\
mpls,in\_port=4,mpls\_label=32,mpls\_bos=1 \\
actions=pop\_mpls:0x0800,output:1 }
\centering
\caption{Flujo 2}
\label{fig:Flujo2}
\end{figure}

Este flujo toma todo paquete recibido por el puerto openflow n\'umero 4(interfaz nf2), retira el cabezal mpls y finalmente lo reenvía por el puerto n\'umero 1(interfaz eth1).\\

De esta forma todos los paquetes asociados al servicio S3 son transportados desde el nodo de ingreso \textit{Galois} al nodo de egreso \textit{Poisson} mediante un solo enlace, utilizando un solo nivel de etiquetas mpls.\\

Analizando el camino que deben seguir los paquetes que atraviesan la red del laboratorio en el sentido inverso; es decir, desde la subred B hacia la subred A(servicio S4), el camino teórico es el siguiente:

$$<(Poisson, nf_1), (Oz, nf_0)>$$ 

Analizando primero la tabla de flujos del nodo \textit{Poisson}, el primer salto del camino es implementado por los siguientes flujos:

\begin{center}
\textit{cookie=0.0, duration=655.065s, table=0, n\_packets=0, n\_bytes=0, priority=5, \\
ip,in\_port=1, nw\_src=20.20.20.64/26,nw\_dst=20.20.20.128/26 \\
actions=dec\_ttl,push\_mpls:0x8847,set\_field:33->mpls\_label,set\_mpls\_ttl(128), goto\_table:1 \\
cookie=0.0, duration=655.065s, table=0, n\_packets=0, n\_bytes=0, priority=5, \\
mpls,in\_port=1,mpls\_label=33 actions=push\_mpls:0x8847,set\_fied:30->mpls\_label,set\_mpls\_ttl(128),output:3
}
\end{center}

Notar como primera diferencia en comparación al camino anterior, en este caso se tienen dos flujos: un primer flujo para colocar la etiqueta asociada al servicio (etiqueta interna) y un segundo flujo para colocar la etiqueta de reenvío (etiqueta externa). 

El camino anterior carece de etiqueta externa puesto que al ser un camino de un solo salto, el primer nodo coincide con el pen\'ultimo, y al implementar PHP no es necesario colocar etiqueta externa. 

Tras colocar el par de etiquetas mpls sobre el paquete al ingreso, el mismo es reenviado a trav\'es del puerto n\'umero 3(interfaz nf1).\\

El siguiente salto en el camino, es implementado por el siguiente flujo en la tabla del nodo \textit{Oz}:

\begin{center}
\textit{cookie=0.0, duration=610.380s, table=0, n\_packets=243, n\_bytes=25748, priority=5, \\
mpls,in\_port=3,mpls\_label=30,mpls\_bos=0 actions=pop\_mpls:0x8847,output:2 }
\end{center}

Tras ingresar un paquete, se cambia el valor de la etiqueta externa para luego reenviarse por el puerto n\'umero 4(interfaz nf2).\\

Luego el tramo final del camino es implementado en el nodo \textit{Galois} mediante el siguiente flujo:

\begin{center}
\textit{cookie=0.0, duration=531.341s, table=0, n\_packets=234, n\_bytes=23868, priority=5, \\
mpls,in\_port=2,mpls\_label=33,mpls\_bos=1 actions=pop\_mpls:0x0800,output:1 }
\end{center}

Análogamente el lector puede completar el análisis de la correspondencia entre los caminos te\'oricos y los calculados por el algoritmo de ruteo. A continuaci\'on se analiza el algoritmo de distribucion de etiquetas.

\subsubsection{Verificaci\'on de Algoritmo de distribución de etiquetas}

[De repente comentar en funcion al escenario anterior porque anda bien]

En la siguiente secci\'on se analiza la clasificaci\'on de tr\'afico.

\subsubsection{Clasificaci\'on de tr\'afico}
Se implementa clasificaci\'on de tr\'afico en el nodo ingreso para un servicio en particular. En el escenario definido, se realiza clasificaci\'on de tr\'afico basándose en la numeraci\'on IP origen y destino de un paquete. De esta forma, por ejemplo el nodo Galois determina que camino debe tomar un paquete con origen en la Subred A y destino la Subred B. Luego en cada nodo intermedio los paquetes son procesados acorde a las reglas de reenvío en base a la conmutación de etiquetas mpls.

Para verificar el correcto funcionamiento de cada flujo OpenFlow involucrado en la clasificaci\'on de tr\'afico, as\'i como el posterior procesamiento dentro de la red del prototipo, se elige un servicio y se observa el procesamiento de los paquetes asociados en cada nodo involucrado en el camino.\\

\begin{figure}[ht!] 
\centering    
\includegraphics[width=1.0\textwidth]{LabE1P1CaputrasTrafico0}
\caption[Capturas de tr\'afico con tcpdump - servicio S1]{Capturas de tr\'afico con tcpdump - servicio S1}
\label{fig:LabE1P1CapsTraf}
\end{figure}

En la figura \ref{fig:LabE1P1CapsTraf} se muestra el procesamiento de los paquetes asociados al servicio S1 en cada uno de los nodos que componen al LSP, generando tr\'afico desde un host en la sub red A con destino a otro host en la subred C.

Como puede observarse en el primer cuarto de la imagen (cuarto superior izquierdo), el cual se corresponde con una captura hecha con el comando \textbf{tcpdump} en la interfaz \textbf{eth1} del nodo \textit{Galois}, se reciben paquetes ICMP request con origen 20.20.20.131 y destino 20.20.20.05. Tambi\'en puede observarse en esta imagen paquetes ICMP reply a los paquetes request enviados.

Luego, como se muestra en el segundo cuarto de la imagen(cuarto superior derecho), el cual se corresponde con una captura realizada en la interfaz nf0 de dicho nodo, a cada paquete ICMP request recibido por la interfaz eth1 se le coloca un cabezal mpls con la etiqueta 31 y se reenvía por la interfaz en cuestión. Finalmente al arribar al nodo \textit{Oz} por la interfaz \textbf{nf0} (cuarto inferior izquierdo de la imagen), el cabezal mpls es retirado de cada paquete para luego ser reenviado por la interfaz \textbf{eth1} hacia la subred C (cuarto inferior derecho de la imagen).\\

En la figura \ref{fig:LabE1P1CapHost} se muestra una captura de pantalla del comando ping utilizado para generar tr\'afico ICMP desde el host 20.20.20.131 en la subred A, al host 20.20.20.05 en la subred C.

\begin{figure}[h!] 
\centering    
\includegraphics[width=0.6\textwidth]{E1P1202020131-2020205}
\caption[Capturas de comando ping H1-Subred A]{Capturas de comando ping H1-Subred A}
\label{fig:LabE1P1CapHost}
\end{figure}

Análogamente se procesan los paquetes asociados a los restantes servicios en el sistema. Para complementar el ejemplo anterior, en la figura \ref{fig:LabE1P1CapsTraf2} se muestra el procesamiento de los paquetes asociados al servicio S3, utilizando el comando ping para generar tr\'afico desde el host 20.20.20.131 en la subred A, al host 20.20.20.67 en la subred B. En la imagen se muestran las capturas de tr\'afico utilizando el comando tcpdump para las interfaces eth1 (cuarto superior izquierdo de la imagen) y nf2 (cuarto superior derecho de la imagen) en el nodo Galois, y las interfaces nf2 (cuarto inferior izquierdo) y eth1 (cuarto inferior derecho) en el nodo Poisson.\\

En la siguiente secci\'on se discute el comportamiento del prototipo, cuando la topolog\'ia cambia por ejemplo por una falla t\'ecnica en un enlace, utilizando el escenario y los servicios definidos anteriormente.

\begin{figure}[ht!] 
\centering    
\includegraphics[width=1.0\textwidth]{LabE1P1CaputrasTrafico}
\caption[Capturas de tr\'afico con tcpdump - servicio S3]{Capturas de tr\'afico con tcpdump - servicio S3}
\label{fig:LabE1P1CapsTraf2}
\end{figure}

%Sin embargo RAUFlow admite realizar clasificaci\'on de tr\'afico por un conjunto bastante m\'as amplio de atributos. A continuaci\'on se definen una nueva lista de servicios orientados a demostrar la correcta implementaci\'on de esta funcionalidad.

%Cabe destacar que para esta prueba los servicios anteriormente creados son eliminados.

%\begin{table}[h]
%\begin{tabular}{| l | l | l | p{4cm} | p{4cm} |}
%\hline
%Nombre & Ingreso & Egreso & Clasificación & Descripción \\ \hline
%
%\crule[Aquamarine]{0.3cm}{0.3cm} S1 & Galois - eth1 & Oz - eth1 & ip\_src=20.20.20.128/26 ip\_dst=20.20.20.0/26 & Tr\'afico de Subred A a Subred C \\ \hline
%
%\crule[Red]{0.3cm}{0.3cm} S2 & Oz - eth1 & Galois - eth1 & ip\_src=20.20.20.0/26 ip\_dst=20.20.20.128/26 & Tr\'afico de Subred C a Subred A \\ \hline
%
%\crule[ForestGreen]{0.3cm}{0.3cm} S3 & Galois - eth1 & Poisson - eth1 & ip\_src=20.20.20.128/26 ip\_dst=20.20.20.64/26 & Tr\'afico de Subred A a Subred B \\ \hline
%
%\crule[LimeGreen]{0.3cm}{0.3cm} S4 & Poisson - eth1 & Galois - eth1 & ip\_src=20.20.20.64/26 ip\_dst=20.20.20.128/26 & Tr\'afico de Subred B a Subred A \\ \hline
%
%\crule[RoyalPurple]{0.3cm}{0.3cm} S5 & Poisson - eth1 & Oz - eth1 & ip\_src=20.20.20.64/26 ip\_dst=20.20.20.0/26 & Tr\'afico de Subred B a Subred C \\ \hline
%
%\crule[YellowOrange]{0.3cm}{0.3cm} S6 & Oz - eth1 & Poisson - eth1 & ip\_src=20.20.20.0/26 ip\_dst=20.20.20.64/26 & Tr\'afico de Subred C a Subred B \\ \hline
%\end{tabular}
%\vspace{0.3cm}

%\caption[CU1 - Escenario 1, servicios extra]{CU1 - Escenario 1, servicios extra}
%\label{table:TablaFlujos2}
%\end{table}

\newpage
\subsubsection{Actualizaci\'on de rutas}
Para probar el correcto funcionamiento del algoritmo de ruteo, en la actualizaci\'on de rutas existentes ante un cambio en la topolog\'ia, se trabaja con los servicios previamente definidos en \ref{table:TablaFlujos}.\\ 

Partiendo de este escenario, se simula un desperfecto t\'ecnico en los enlaces \\ $<(Galois, nf_2), (Poisson, nf_2)>$ y  $<(Poisson, nf_2), (Poisson, nf_2)>$ (por ejemplo simplemente desconectando el cable que conecta ambos nodos), provocando de esta forma una actualizaci\'on en la topolog\'ia y la posterior ejecuci\'on de los algoritmos de ruteo y distribución de etiquetas para los nuevos LSPs.\\ 
 
Acorde a los costos de la topolog\'ia (ver figura \ref{fig:LaboratorioDePruebasCostos}), en la nueva topolog\'ia los caminos te\'oricos asociados a cada servicio quedan de la siguiente forma:

\newpage
\begin{figure}[ht!] 
\centering    
\includegraphics[width=1.0\textwidth]{CU1P1CaminosRecalculo}
\caption[Escenario 1 - Caminos para servicios recalculados]{Escenario 1 - Caminos para servicios recalculados}
\label{fig:CUP1Caminos2}
\end{figure}
 
Notese que los \'unicos caminos que cambian son los asociados a los servicios S3 y S6. Cabe destacar adem\'as que en la nueva topolog\'ia existe m\'as de un camino de m\'inimo costo para el servicio S3; los caminos $<(Galois, nf0), (Oz, nf1)>$ y $<(Galois, nf0), (Oz, nf2), (Alice, nf0)>$ presentan ambos el costo 4. Por tanto se tienen dos resultados v\'alidos posibles para la salida del algoritmo de ruteo para este servicio.\\

Para comprobar que el algoritmo de ruteo recaclula correctamente las rutas, se inspeccionan nuevamente las tablas de flujos de cada nodo en el laboratorio (ver figuras \ref{fig:CU1P2DumpFlows1}-\ref{fig:CU1P2DumpFlows4}), comparando los caminos calculados con los caminos te\'oricos.\\

\begin{figure}[h] 
\centering    
\includegraphics[width=1.0\textwidth]{LabE1P2Al}
\caption[Tabla de flujos ovs - Alice]{Tabla de flujos ovs - Alice}
\label{fig:CU1P2DumpFlows1}
\end{figure}

\newpage
\begin{figure}[h] 
\centering    
\includegraphics[width=1.0\textwidth]{LabE1P2Gal}
\caption[Tabla de flujos ovs - Galois]{Tabla de flujos ovs - Galois}
\label{fig:CU1P2DumpFlows2}
\end{figure}

\begin{figure}[h] 
\centering    
\includegraphics[width=1.0\textwidth]{LabE1P2Poi}
\caption[Tabla de flujos ovs - Poisson]{Tabla de flujos ovs - Poisson}
\label{fig:CU1P2DumpFlows3}
\end{figure}

\newpage
\begin{figure}[ht!] 
\centering    
\includegraphics[width=1.0\textwidth]{LabE1P2Oz}
\caption[Tabla de flujos ovs - Oz]{Tabla de flujos ovs - Oz}
\label{fig:CU1P2DumpFlows4}
\end{figure}

Tomando como ejemplo la actualizaci\'on del LSP asociado al servicio S3, mientras que en la topolog\'ia original el camino asociado es $<(Galois, nf2)>$, tras la actualizaci\'on de la topolog\'ia el camino correcto puede ser o bien $<(Galois, nf0),(Oz, nf1)>$ o bien \\ $<(Galois, nf0), (Oz, nf2), (Alice, nf0)>$.\\

Al cambiar el camino, los flujos asociados a cada nodo en el camino tambi\'en deben cambiar. En particular como el camino nuevo no comparte ning\'un salto con el camino original, los flujos asociados al camino viejo deben ser eliminados de cada nodo.

Recordando la tabla de flujos original del nodo \textit{Galois} (ver imagen \ref{fig:CU1P1DumpFlows1}), el flujo \ref{fig:Flujo1} es utilizado para procesar y reenviar paquetes al nodo Poisson. En la tabla de flujos actualizada, este flujo es remplazado por el siguiente flujo (ver imagen \ref{fig:CU1P2DumpFlows2}):

\begin{center}
\textit{cookie=0.0, duration=192.239s, table=0, n\_packets=197, n\_bytes=19306, priority=5, \\
ip,in\_port=1, nw\_src=20.20.20.128/26,nw\_dst=20.20.20.64/26 \\
actions=dec\_ttl,push\_mpls:0x8847,set\_field:32->mpls\_label,set\_mpls\_ttl(128), goto\_table:1 \\
cookie=0.0, duration=197.238s, table=0, n\_packets=197, n\_bytes=19306, priority=5, \\
mpls,in\_port=1,mpls\_label=32 actions=push\_mpls:0x8847,set\_fied:32->mpls\_label,set\_mpls\_ttl(128),output:2}
\end{center}

Mediante este par de flujos, se les colocan dos cabezales mpls a los paquetes asociados al servicio, para luego ser reenviados por el puerto numero 2 (interfaz nf0) al nodo \textit{Oz}.\\

Por otra parte en la tabla de flujos del nodo \textit{Oz}, se incorpora el siguiente flujo:

\begin{center}
\textit{cookie=0.0, duration=280.155s, table=0, n\_packets=280, n\_bytes=29680, priority=5, \\
mpls,in\_port=2,mpls\_label=32,mpls\_bos=0 actions=set\_field:30->mpls\_label,dec\_mpls\_ttl,output:4 }
\end{center}

El mismo implementa el cambio de etiqueta mpls en el paquete, y su posterior reenvio por el puerto n\'umero 4 (interfaz nf2).\\

Luego en la tabla de flujos del nodo \textit{Alice} se incorpora el siguiente flujo:

\begin{center}
\textit{cookie=0.0, duration=48.510s, table=0, n\_packets=48, n\_bytes=5088, priority=5, \\
mpls,in\_port=2,mpls\_label=32,mpls\_bos=0 actions=pop\_mpls:0x8847,output:2 }
\end{center}

Este flujo implementa el pop de la etiqueta mpls externa en el penultimo nodo del LSP (penultimate-pop-hoping). Tras realizar esta acci\'on reenvia el paquete por el puerto n\'umero 2 (interfaz nf0).\\

Finalmente cuando el paquete arriba al nodo \textit{Poisson}, el procesamiento final del paquete, realizado con anterioridad por el flujo \ref{fig:Flujo2} ahora se realiza mediante el siguiente flujo: 

\begin{center}
\textit{cookie=0.0, duration=320.758s, table=0, n\_packets=320, n\_bytes=32740, priority=5, \\
mpls,in\_port=2,mpls\_label=32,mpls\_bos=1 actions=pop\_mpls:0x0800,output:1 }
\end{center}

En la figura \ref{fig:CU1P1DumpFlows1} se muestran capturas tcpdump de las interfaces por las cuales el tr\'afico asociado al servicio S3 atraviesa la red del laboratorio en el nuevo camino calculado.

Enumerando las figuras de izquierda a derecha y de arriba hacia abajo, la primer imagen se corresponde con una captura sobre la interfaz eth1 del nodo Galois, la segunda con la interfaz nf0 del mismo nodo, la tercera con la interfaz nf2 del nodo Oz, la cuarta con la interfaz nf0 del nodo Alice y finalmente la quinta con la interfaz eth1 en el nodo Poisson.

\newpage
\begin{figure}[ht!] 
\centering    
\includegraphics[width=1.0\textwidth]{LabE1P2SnapshotTrafico1}
\caption[Tabla de flujos ovs - Oz]{Capturas de tr\'afico con tcpdump para el servicio S3 tras actualización topol\'ogica}
\label{fig:LabE1P1CapsTraf3}
\end{figure}

Análogamente se puede estudiar la correspondencia entre el camino te\'orico y el camino calculado para el servicio S6.\\
 
De esta forma se verifica para el laboratorio de pruebas construido, la correctitud del algoritmo de ruteo, el algoritmo de distribución de etiquetas, se prueba el correcto funcionamiento de las tablas de flujos Open vSwitch y los flujos particulares utilizados para implementar clasificacion de tr\'afico, se comprueba el correcto accionar del programa ante cambios en la topolog\'ia (recalculo de LSPs), y finalmente se valida la construcción de un servicio de rede privada multipunto de capa 3.

En la siguiente seccci\'on se discute la verificacion de otros aspectos importantes en el funcionamiento del prototipo, utilizando como escenario de pruebas la construcci\'on de dos servicios de red privada punto a punto de capa 3, utilizando dos organizaciones diferentes.   

\subsection{Escenario 2}

Este escenario representa una red privada punto a punto de capa 3. Esta compuesto por dos organizaciones diferentes, cada una de ellas con dos sucursales f\'isicamente separadas. Adem\'as, ambas organizaciones utilizan el mismo espacio de direccionamiento IP para sus repectivas subredes.\\

Vale la pena mencionar que debido a las limitaciones impuestas por el hardware disponible; recordar que cada nodo en el prototipo cuenta con una \'unica interfaz f\'isica disponible para conectarse a la subred de un “cliente”, solo se pueden conectar cuatro subredes al prototipo. De esta forma, el \'unico  escenario para el cual se tienen al menos dos clientes diferentes con al menos dos oficinas físicamente separadas, es el caso m\'as simple en donde cada organización tiene dos subredes. 

Este escenario se traduce en dos servicios de VPN punto a punto. Sin embargo los aspectos de implementaci\'on que en esta secci\'on se pretenden verificar, son independientes de si el servicio de VPN es punto a punto o multipunto.

\begin{figure}[h] 
\centering    
\includegraphics[width=1.0\textwidth]{CU1P2}
\caption[VPN de capa 3 - Escenario 2]{VPN de capa 3 - Escenario 2}
\label{fig:CUP2}
\end{figure}

Sobre este escenario se busca validar los siguientes aspectos en relaci\'on a la implementaci\'on:

\begin{enumerate}
\item El prototipo soporta la construcci\'on de servicios de VPN de capa 3 para multiples organizaciones (recordar que en el escenario anterior se trabaja con una \'unica organización)
\item La implementaci\'on utilizada en el prototipo para transportar tr\'afico de una organización desde un nodo de ingreso a un nodo de egreso, es independiente de la numeraci\'on de capa 3 de dicho tr\'afico. En otras palabras soporta organizaciones con numeraci\'on IP solapadas.
\end{enumerate}

Para la construcci\'on de este escenario se instancian los siguientes servicios en el sistema (ver tabla). Por cada red privada se instancian dos servicios (uno para cada sentido del tr\'afico).

\begin{table}[h]
\begin{tabular}{| l | l | l | p{4cm} | p{4cm} |}
\hline
Nombre & Ingreso & Egreso & Clasificación & Descripción \\ \hline

\crule[Aquamarine]{0.3cm}{0.3cm} S1 & Galois - eth1 & Alice - eth1 & ip\_src=20.20.20.64/26 ip\_dst=20.20.20.0/26 & Tr\'afico de Subred A a Subred B \\ \hline

\crule[Red]{0.3cm}{0.3cm} S2 & Alice - eth1 & Galois - eth1 & ip\_src=20.20.20.0/26 ip\_dst=20.20.20.64/26 & Tr\'afico de Subred B a Subred A \\ \hline

\crule[ForestGreen]{0.3cm}{0.3cm} S3 & Poisson - eth1 & Oz - eth1 & ip\_src=20.20.20.64/26 ip\_dst=20.20.20.0/26 & Tr\'afico de Subred A' a Subred B' \\ \hline

\crule[LimeGreen]{0.3cm}{0.3cm} S4 & Oz - eth1 & Poisson - eth1 & ip\_src=20.20.20.0/26 ip\_dst=20.20.20.64/26 & Tr\'afico de Subred B' a Subred A' \\ \hline

\end{tabular}
\vspace{0.3cm}
\caption[CU1 - Escenario 2]{CU1 - Escenario 2}
\label{table:TablaFlujos3}
\end{table}

Para todos estos servicios adem\'as se indica como valor de ethertype 0x0800, correspondiente al tipo de tr\'afico IPv4.

\newpage
\begin{figure}[ht!] 
\centering    
\includegraphics[width=1.0\textwidth]{E2P1Gal}
\caption[Tabla de flujos ovs - Galois]{Tabla de flujos ovs - Galois}
\label{fig:CU1P1DumpFlows1}
\end{figure}

\begin{figure}[h!] 
\centering    
\includegraphics[width=1.0\textwidth]{E2P1Oz}
\caption[Tabla de flujos ovs - Oz]{Tabla de flujos ovs - Oz}
\label{fig:CU1P1DumpFlows2}
\end{figure}

\begin{figure}[h!] 
\centering    
\includegraphics[width=1.0\textwidth]{E2P1Poi}
\caption[Tabla de flujos ovs - Poisson]{Tabla de flujos ovs - Poisson}
\label{fig:CU1P1DumpFlows3}
\end{figure}

\begin{figure}[h!] 
\centering    
\includegraphics[width=1.0\textwidth]{E2P1Al}
\caption[Tabla de flujos ovs - Alice]{Tabla de flujos ovs - Alice}
\label{fig:CU1P1DumpFlows4}
\end{figure}

llalala

\newpage
\section{VPN de capa 2}

A diferencia de las VPN de capa 3, las cuales se definen para un protocolo o aplicaci\'on particular (por ejemplo IPv4), las redes privadas de capa 2 permiten definir servicios de forma agnóstica a los mismos, transportando tr\'afico entre dos subredes de cualquier protocolo de capa 3 o aplicación.

En cierta forma, puede pensarse en un servicio de VPN de capa 2 entre dos subredes por ejemplo, como un “cable” virtual que los conecta directamente.\\

En este trabajo, a modo de ejemplo se implementan dos servicios de VPN de capa 2, sobre un escenario minimista en el que se cuenta con una \'unica organización compuesta por dos sucursales físicamente separadas, sobre la red del laboratorio previamente definida.

De esta forma lo que se busca con este escenario es construir dos “cable” y verificar que se transporta correctamente tr\'afico desde una subred a otra, independientemente del protocolo de capa 3 por ejemplo.

\subsection{Escenario}
El escenario implementado para este caso de uso se muestra en la figura \ref{fig:CUP3} . El mismo se compone de una organización con dos sucursales físicamente separadas. Se instancia dos servicios (uno para cada sentido del tr\'afico), para los cuales a su vez se muestra en la figura los caminos calculados por el algoritmo de ruteo.

\newpage
\begin{figure}[h] 
\centering    
\includegraphics[width=1.0\textwidth]{CU1P2}
\caption[VPN de capa 2]{VPN de capa 2}
\label{fig:CUP3}
\end{figure}

En la siguiente tabla (ver tabla \ref{table:TablaFlujos4}) se muestran los servicios instanciados.

\begin{table}[h]
\begin{tabular}{| l | l | l | p{4cm} | p{4cm} |}
\hline
Nombre & Ingreso & Egreso & Clasificación & Descripción \\ \hline

\crule[Aquamarine]{0.3cm}{0.3cm} S1 & Galois - eth1 & Poisson - eth1 & - & Tr\'afico de capa 2 de Subred A a Subred B \\ \hline

\crule[Red]{0.3cm}{0.3cm} S2 & Poisson - eth1 & Galois - eth1 & - & Tr\'afico de capa 2 de Subred B a Subred A \\ \hline


\end{tabular}
\vspace{0.3cm}
\caption[CU2 - Escenario, servicios creados]{CU2 - Escenario, servicios creados}
\label{table:TablaFlujos4}
\end{table}