% ************************** Thesis Abstract *****************************
% Use `abstract' as an option in the document class to print only the titlepage and the abstract.
\begin{abstract}


Actualmente, desarrollar aplicaciones de red,  manipular el contenido de un paquete e incluso algo tan simple como analizar un cabezal de un paquete es una tarea difícil que requiere de trabajar a un nivel muy bajo y con un nivel de conocimiento bastante específico en el área; y todo esto trabajando en un entorno de software libre. Por otro lado si se quiere hacer esto en un contexto comercial, solo es posible si el equipo o API de funcionalidades disponible así lo permite. Este escenario no hace otra cosa que entorpecer y ralentizar el desarrollo de nuevos protocolos y servicios, así como la innovación en el área. Las Redes Definidas por Software (SDN) desacopla los planos de control y datos a la vez que estandariza la forma en que cualquier equipo de red puede ser manipulado y así también el tráfico que estos procesan. De esta forma se generan las condiciones para que cualquier equipo pueda ser utilizado de forma transparente en la definición de nuevos servicios y protocolos, independientemente del fabricante; facilitando así la innovación en el área. 

En este proyecto se construye un prototipo de red de backbone MPLS utilizando el enfoque SDN y placas de red aceleradas en hardware reconfigurable (NetFPGA), orientado a la provisión de redes privadas virtuales como servicio; de cara a lo que podría ser una posible implementación de la nueva red académica RAU2. Se diseña e implementa un equipo enrutador de red denominado RAU-Switch, compatible con las tecnologías IP, MPLS y OpenFlow, orientado a convivir con equipos legados en un futuro , y a un bajo costo económico. Ademas se desarrolla un conjunto de aplicaciones de control y gestión de red denominado RAUFlow y se contribuye a la comunidad en el desarrollo de documentación exhaustiva en relación a tecnologías muy recientes de las que algunas poco se conoce y otras aun están en fases experimentales.

\textbf{Palabras clave:} Redes de Computadoras, SDN, OpenFlow, NetFPGA, MPLS, Open vSwitch.
\end{abstract}



