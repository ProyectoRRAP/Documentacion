%%%%%%%%%%%%%%%%%%%%%%%%%%%%%%%%%%%%%%%%%
% Beamer Presentation
% LaTeX Template
% Version 1.0 (10/11/12)
%
% This template has been downloaded from:
% http://www.LaTeXTemplates.com
%
% License:
% CC BY-NC-SA 3.0 (http://creativecommons.org/licenses/by-nc-sa/3.0/)
%
%%%%%%%%%%%%%%%%%%%%%%%%%%%%%%%%%%%%%%%%%

%----------------------------------------------------------------------------------------
%	PACKAGES AND THEMES
%----------------------------------------------------------------------------------------

\documentclass{beamer}

\mode<presentation> {

% The Beamer class comes with a number of default slide themes
% which change the colors and layouts of slides. Below this is a list
% of all the themes, uncomment each in turn to see what they look like.

%\usetheme{default}
%\usetheme{AnnArbor}
%\usetheme{Antibes}
%\usetheme{Bergen}
%\usetheme{Berkeley}
%\usetheme{Berlin}
%\usetheme{Boadilla}
%\usetheme{CambridgeUS}
%\usetheme{Copenhagen} --
%\usetheme{Darmstadt} --
%\usetheme{Dresden}
%\usetheme{Frankfurt} --
%\usetheme{Goettingen}
%\usetheme{Hannover}
%\usetheme{Ilmenau}
%\usetheme{JuanLesPins}
%\usetheme{Luebeck}
\usetheme{Madrid}
%\usetheme{Malmoe}
%\usetheme{Marburg}
%\usetheme{Montpellier}
%\usetheme{PaloAlto}
%\usetheme{Pittsburgh}
%\usetheme{Rochester}
%\usetheme{Singapore}
%\usetheme{Szeged}
%\usetheme{Warsaw}

% As well as themes, the Beamer class has a number of color themes
% for any slide theme. Uncomment each of these in turn to see how it
% changes the colors of your current slide theme.

%\usecolortheme{albatross}
%\usecolortheme{beaver}
%\usecolortheme{beetle}
%\usecolortheme{crane}
%\usecolortheme{dolphin}
%\usecolortheme{dove}
%\usecolortheme{fly}
%\usecolortheme{lily}
%\usecolortheme{orchid}
%\usecolortheme{rose}
%\usecolortheme{seagull}
%\usecolortheme{seahorse}
%\usecolortheme{whale}
%\usecolortheme{wolverine}

%\setbeamertemplate{footline} % To remove the footer line in all slides uncomment this line
%\setbeamertemplate{footline}[page number] % To replace the footer line in all slides with a simple slide count uncomment this line

%\setbeamertemplate{navigation symbols}{} % To remove the navigation symbols from the bottom of all slides uncomment this line
}

\usepackage[spanish]{babel}
\usepackage[utf8]{inputenc}

\usepackage{graphicx} % Allows including images
\usepackage{booktabs} % Allows the use of \toprule, \midrule and \bottomrule in tables

\usepackage{subcaption}
\captionsetup{compatibility=false}

\renewcommand\footnotemark{}
\renewcommand\footnoterule{}

\usepackage[export]{adjustbox}
\usepackage{wrapfig}


\newenvironment{figure*}%
{\begin{figure}}
{\end{figure}}

%----------------------------------------------------------------------------------------
%	TITLE PAGE
%----------------------------------------------------------------------------------------

\title[RRAP]{Proyecto de grado: Routers Reconfigurables de Altas Prestaciones} % The short title appears at the bottom of every slide, the full title is only on the title page

\author{Rodrigo Amaro, Emiliano Viotti}
\institute[UdelaR] % Your institution as it will appear on the bottom of every slide, may be shorthand to save space
{
Instituto de Computaci\'on \\ Facultad de Ingeniería \\ Universidad de la República \\ \vspace{0.2cm} Tutores: Dr. Eduardo Gramp\'in, MSc. Mart\'in Giachino % Your institution for the title page
%\medskip
%\textit{john@smith.com} % Your email address
}
\date{\today} % Date, can be changed to a custom date

\begin{document}

\begin{frame}
\titlepage % Print the title page as the first slide
\end{frame}


\begin{frame}
\frametitle{Agenda} % Table of contents slide, comment this block out to remove it
\tableofcontents % Throughout your presentation, if you choose to use \section{} and \subsection{} commands, these will automatically be printed on this slide as an overview of your presentation
\end{frame}

%----------------------------------------------------------------------------------------
%	PRESENTATION SLIDES
%----------------------------------------------------------------------------------------


%------------------------------------------------
\section{Motivaci\'on} 

\begin{frame}
\frametitle{Motivaci\'on} 

\begin{block}{Redes acad\'emicas}
Internet no parece apropiada para su utilizaci\'on en el contexto académico en el desarrollo de nuevos protocolos y servicios, investigaci\'on y la innovaci\'on en el \'area.
\end{block}

\begin{figure}[h] 
\centering    
\includegraphics[width=0.7\textwidth]{imagenes/redesAcademicas.png}
\label{fig:AcademicNetworks}
\end{figure}

\end{frame}

\begin{frame}
\frametitle{Motivaci\'on} 

\begin{block}{Red Académica Uruguaya (RAU)}
A nivel local, la RAU es un emprendimiento de la Universidad de la República administrado por el SeCIU con los objetivos de unir a las instituciones académicas nacionales en una red de alcance nacional y a trav\'es de ella conectarlas a Latinoam\'erica. 
\end{block}

\begin{block}{RAU2}
Remplazo de la actual red académica, es una red avanzada de altas prestaciones que estar\'ia dotada de funciones de virtualizaci\'on de redes flexibles en su definici\'on y uso.
\end{block}

\begin{figure}[h] 
\centering    
\includegraphics[width=0.3\textwidth]{imagenes/logorau2.png}
\label{fig:RAU}
\end{figure}

\end{frame}

\begin{frame}
\frametitle{Motivaci\'on} 

\begin{block}{Hardware comercial}
Los equipos de red de backbone comerciales como HP, CISCO, Juniper son costos y generalmente de naturaleza cerrada. Las funcionalidades del hardware se restringen a las funcionalidades expuesta por
una API propietaria. 
\end{block}

\begin{figure}[htp]
\centering
\includegraphics[width=.25\textwidth]{imagenes/corerouter2.png}\hfill
\includegraphics[width=.35\textwidth]{imagenes/corerouter1.png}\hfill
\includegraphics[width=.35\textwidth]{imagenes/corerouter3.jpg}
\label{fig:figure3}
\end{figure}
\end{frame}
%------------------------------------------------

%------------------------------------------------
\section{Definición del problema} 


\begin{frame}
\frametitle{Definición del problema} 

\begin{block}{Definición del Problema }
Definir el problema de forma breve y concisa
\end{block}
\end{frame}

\begin{frame}
\frametitle{Definición del problema} 

\begin{block}{Objetivos}
Enumerar los principales objetivos
\end{block}

\begin{block}{Resultados esperados}
Enumerar los principales resultados esperados
\end{block}

\end{frame}

%------------------------------------------------

%------------------------------------------------
\section{Conceptos preliminares} 


\begin{frame}
\frametitle{SDN-Enfoque tradicional} 


SDN Es un enfoque arquitectonico alternativo al enfoque tradicional de redes.

Enfoque tradicional:

PHOTO

\begin{block}{Enfoque tradicional}
La inteligencia y estado de la red se encuentra distribuida en los mismos dispositivos que reenvian la informacion
\end{block}
%Imagen enfoque tradicional


\end{frame}

\begin{frame}
\frametitle{SDN-Plano de control y plano de datos} 


\begin{block}{Plano de control}

El plano de control donde reside la inteligencia de la red... robar de CRA
\end{block}
ejemplos son: algoritmos de ruteo tradicionales, OSPF, RIP,Firewalls


\begin{block}{Plano de datos}
\end{block}



%Imagen enfoque tradicional

\end{frame}

\begin{frame}
\frametitle{SDN} 
Definicion:
Y mostrar imagen enfoque tradicional vs sdn

\end{frame}


%SDN 
%
%PUNTO DE CONTROL CENTRALIZADO DE LA RED
%BUSCAR OTROS EJEMPLOS DE implementaciones SDN
%
%EJEMPLOS DE PLANO DE CONTROL, ospfd, firewalls etc
%
%
%Alternativa al enfoque tradicional
%EN SDN MOSTRAR UNA IMAGEN DE ARQUITECTURA ACTUAL Y 
%ARQUITECTURA SDN
%Pueden ser parecidas a esto:
%http://millennialmainframer.com/2012/09/the-mainframe-and-software-defined-networking/
%http://www.aryaka.com/why-sdn-concepts-need-to-extend-into-the-wan/
%
%
%Es una solucion abierta de SDN
%
%Define una arquitectura
%
%
%-----Elegir una imagen que muestre las apps arriba del controler
% y explicar como se programa------
%COntrolador
%(Interfaz sur
%Interfaz norte
%Interfaz este-oeste)
%Protocolo
%Dispositivo opewnflow-capable(Canal openflow, tablas)
%----------------------
%
%Define un protocolo y una estructura de mensaje
%
%
%Define un protocolo de red que permite comunicar el controlador(plano de control) 
%con los dispositivos (plano de datos)
%Rapido hablar de los tres tipos de mensajes(Controlador a disp,asinc, simetricos(keepalives,echos))
%
%Define la estructura de la tabla de flujos
%
%(Match field)
%Que match fields? Mostrar una imagen con ether, mpls tpc ip
%
%(Contador, estadisticas por flujo)
%Instrucciones(Que hacer? reenviar ,modificar y reenviar, broadcast, multicast, normal).
%
%
%Ventajas y desventajas?



\begin{frame}
\frametitle{OpenFlow} 

\end{frame}

\begin{frame}
\frametitle{VPN} 

\end{frame}

\begin{frame}
\frametitle{MPLS} 

\end{frame}

%------------------------------------------------
\section{Arquitectura de la soluci\'on} 

\begin{frame}
\frametitle{Arquitectura de la soluci\'on} 

Poner una imagen de la arquitectura esperada basada en SDN
plano de control, plano de datos dispositivos OpenFlow capaz, etc. Mi idea es ahi explicar
las cuestiones que hay que atacar. En el trabajo hay que implementar:
1) Plano de Control
2) Dispisitivos SDN/OpenFlow

Redondear ambas lineas de trabajo mientras se explica asi se pasa en las siguientes ppt a hablar de eso

\end{frame}

\begin{frame}
\frametitle{Arquitectura de la soluci\'on} 

\begin{minipage}{0.60\textwidth}
Switch OpenFlow

Dos alternativas: Mencionarlas (NetFPGA + Openflow, ...)

Mostrar las dos fotos del stack Plano Control - Dispositivo con los building blocks

\end{minipage}
\hfill
\begin{minipage}{0.30\textwidth}
\begin{figure}[H]
\raggedright
\includegraphics[width=1.0\textwidth, right]{imagenes/BuilidngBlocks1.png}
\end{figure}
\end{minipage} 



\end{frame}

\begin{frame}
\frametitle{Arquitectura de la soluci\'on} 

Controlador

Utilizamos Ryu blablabla....

Mostrar las dos fotos del stack Plano Control - Dispositivo con los building blocks
\end{frame}

\begin{frame}
\frametitle{Arquitectura de la soluci\'on} 

\onslide<1->{Algoritmo de ruteo}

\onslide<2->{OSPF para el descubrimiento de la topologia y algoritmo de ruteo centralizado

Mostrar las dos fotos del stack Plano Control - Dispositivo con los building blocks}


\end{frame}

\begin{frame}
\frametitle{Arquitectura de la soluci\'on} 

Mapeo de puertos a direcciones IP

SNMP

Mostrar las dos fotos del stack Plano Control - Dispositivo con los building blocks
\end{frame}

\begin{frame}
\frametitle{Arquitectura de la soluci\'on} 

Esbozo de arquitectura final

misma imagen que la inicial pero con los building blocks
\end{frame}

%------------------------------------------------

%------------------------------------------------
\section{Implementaci\'on} 

\begin{frame}
\frametitle{Implementaci\'on} 

\end{frame}
%------------------------------------------------

%------------------------------------------------
\section{Conclusiones} 

\begin{frame}
\frametitle{Conclusiones} 

\end{frame}
%------------------------------------------------

%------------------------------------------------
\section{Trabajo a futuro} 

\begin{frame}
\frametitle{Trabajo a futuro} 

\end{frame}
%------------------------------------------------

%----------------------------------------------------------------------------------------

\end{document} 